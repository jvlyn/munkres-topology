\documentclass{article}
\usepackage[utf8]{inputenc}
\usepackage{amsmath}
\usepackage{amssymb}
\usepackage{setspace}
\usepackage{MnSymbol}
\usepackage{changepage}
\usepackage{graphicx}
\usepackage{gensymb}
\usepackage{fancyhdr}
\usepackage{polynom}
\usepackage{tikz}
\usepackage{tikz-cd}
\graphicspath{ {images/} }
\usepackage{multicol}
\usepackage[letter, portrait, margin=1in]{geometry}
\newcommand{\q}[1]{\overline{#1}}
\newcommand{\nline}{\vspace*{0.5\baselineskip}}
\renewcommand{\arraystretch}{1.2}
\definecolor{grey}{gray}{0.4}
\newcommand{\com}[1]{\textcolor{grey}{#1}}
\usepackage{amsthm}
\usepackage{enumitem}
\newenvironment{solution}
    {\\ [0.5\baselineskip]
    \begin{adjustwidth}{1cm}{}
    \textit{Solution.}
    }
    {\end{adjustwidth}
    \\ [0.5\baselineskip]
    }

\theoremstyle{definition}
\newtheorem{theorem}{Theorem}[subsection]
\newtheorem{corollary}{Corollary}[subsection]
\newtheorem{lemma}{Lemma}[subsection]
\newtheorem{proposition}{Proposition}[subsection]
\newtheorem*{definition}{definition}

\pagestyle{fancy}
\fancyhf{}
\lhead{Topology - Munkres}
\chead{MATH 5345H}
\rhead{Julie Yuan}
\renewcommand\headheight{26pt}
\cfoot{\thepage}

\usepackage{hyperref}

\begin{document}
\begin{flushleft}

\setcounter{section}{1}

\section{Topological Spaces and Continuous Functions}

\subsection{Topological Spaces}

A \textbf{topology} on a set $X$ is a collection $\mathcall{T}$ of subsets of $X$ having the following properties:

\begin{enumerate}
    \item $\varnothing$ and $X$ are in $\mathcal{T}$
    \item unions of any elements in $\mathcal{T}$ are also in $\mathcal{T}$
    \item \textit{finite} intersections of any elements in $\mathcal{T}$ are also in $\mathcal{T}$
\end{enumerate}

A set $X$ on which a topology has been defined is a \textbf{topological space} (we write $(X, \mathcal{T})$). If $X$ is a topological space with topology $\mathcal{T}$, we say that $U \subseteq X$ is an open set of $X$ if $U \in \mathcal{T}$.

\nline

Examples:

\begin{itemize}
    \item The \textbf{discrete topology} consists of all subsets of a set $X$.
    \item The \textbf{finite complement topology} $\mathcal{T}_f$ consists of all $U \subseteq X$ such that $X - U$ is either finite or all of $X$ (so that $\varnothing \in \mathcal{T}_f$).
\end{itemize}

Suppose $\mathcal{T}, \mathcal{T}'$ are topologies on $X$. If $\mathcal{T} \subset \mathcal{T}'$, then $\mathcal{T}$ is coarser than $\mathcal{T}'$ (resp. finer). If neither of these is true, then $\mathcal{T}$ and $\mathcal{T}'$ are incomparable. (Think of the gravel analogy).

Examples:

\begin{itemize}
    \item The finite complement topology on $\mathbb{R}$ is coarser than the standard topology.
    \item The Zariski topology on $\mathbb{R}^n$ is coarser than the standard topology.
\end{itemize}

\subsection{Basis for a Topology}

If $X$ is a set, a \textbf{basis} for a topology on $X$ is a collection $\mathcal{B}$ of subsets of $X$ (basis elements) such that

\begin{enumerate}
    \item for each $x \in X$, there exists at least one basis element $B \in \mathcal{B}$ containing $x$
    \item if $x \in B_1 \cap B_2$, then there exists $B_3 \in \mathcal{B}$ such that $x \in B_3$ and $B_3 \subseteq B_1 \cap B_2$
\end{enumerate}

In the topology $\mathcal{T}$ \textbf{generated by $\mathcal{B}$}, a subset $U \subseteq X$ is open in $X$ if for each $x \in U$, there exists a basis element $B \in \mathcal{B}$ such that $x \in B \subseteq U$.

\nline

Sanity check: this does indeed generate a topology, right?

\begin{enumerate}
    \item We have $\varnothing \in \mathcal{T}$ (vacuously) and since for each $x \in X$, there exists some $B \in \mathcal{B}$ containing $x$ and such that $B \subseteq X$ so $X$ is open.
    \item Take $\{U_\alpha\}_{\alpha \in J}$ of elements in $\mathcal{T}$. We want to show that $U = \cup_{\alpha \in J} U_\alpha \in \mathcal{T}$. Suppose we have $x \in U$. Then, $x \in U_\alpha$ for some $\alpha$. Since $U_\alpha$ is open, there exists a basis element $B \in \mathcal{B}$ such that $x \in B \subset U_\alpha$. Then, $x \in B$ and $B \subset U$ so $U$ is open.
    \item Take $U_1, U_2 \in \mathcal{T}$. We want to show that $U_1 \cap U_2 \in \mathcal{T}$. Given $x \in U_1 \cap U_2$, pick $B_1$ such that $x \in B_1 \subset U_1$ and $B_2$ such that $x \in B_2 \subset U_2$. Then, there exists some $B_3$ such that $x \in B_3 \subset B_1 \cap B_2 \subset U_1 \cap U_2$ so $U_1 \cap U_2 \in \mathcal{T}$. We can use induction to repeat this argument for any finite intersection of sets.
\end{enumerate}

Examples:

\begin{itemize}
    \item The set of all circular regions in the plane is a basis for $\mathbb{R}^2$.
    \item The set of all rectangular regions with edges parallel to the coordinate axes is a basis for $\mathbb{R}^2$.
    \item The collection of one point subsets of any set $X$ is a basis for the discrete topology on $X$.
\end{itemize}

\begin{lemma}[13.1, topology is all unions of basis elements]
Let $X$ be a set; let $\mathcal{B}$ be a basis for a topology $\mathcal{T}$ on $X$. Then $\mathcal{T}$ equals the collection of all the unions of elements of $\mathcal{B}$.
\end{lemma}

\begin{proof}
Given a collection of sets in $\mathcal{B}$, they are also in $\mathcal{T}$ \com{(obviously, because the basis \textit{generates} the topology)} so because $\mathcal{T}$ is closed under unions, their union is in $\mathcal{T}$. Conversely, suppose that $U \in \mathcal{T}$. For each $x \in U$, choose an element $B_x \in \mathcal{B}$ such that $x \in B_x \subset U$ \com{(which exists because $U$ is open)}. Then $U_{x \in U} B_x$ so $U$ is a union of basis elements.
\end{proof}

\begin{lemma}[13.2, conditions for a basis]
Let $X$ be a topological space. Suppose that $\mathcal{C}$ is a collection of open sets of $X$ such that for each open set $U \subseteq X$ and for each $x \in U$, there exists an element $C \in \mathcal{C}$ such that $x \in C \subset U$. Then $\mathcal{C}$ is a basis for the topology of $X$. (This gives us a means for obtaining a basis, given a topology.)
\end{lemma}

\begin{proof}
We want to show that $\mathcal{C}$ is a basis. We know that given $x \in X$, since $X$ is itself open, there exists $C \in \mathcal{C}$ such that $x \in C \subset X$. Now, let $x \in C_1 \cap C_2$ where $C_1, C_2 \in \mathcal{C}$. Since $C_1, C_2$ are open, $C_1 \cap C_2$ must also be open \com{(because topologies are closed under finite intersections)}. By hypothesis, there exists some $C_3 \in \mathcal{C}$ such that $x \in C_3 \subset C_1 \cap C_2$.

\nline

We want to show that $\mathcal{T}'$, the topology generated by $\mathcal{C}$, is equivalent to $\mathcal{T}$, the collection of open sets on $X$ \com{(note that this it \textit{not} obvious)}. If $U \in \mathcal{T}$ and $x \in U$, then by hypothesis, there exists some $C \in \mathcal{C}$ such that $x \in C \subset U$. Hence, $U \in \mathcal{T}'$. Conversely, if we have $W \in \mathcal{T}'$, then $W$ is a union of elements in $\mathcal{C}$ \com{(by Lemma 13.1)}. Since each $C \in \mathcal{C}$ is also an element of $\mathcal{T}$ ($\mathcal{C}$ is a collection of \textit{open sets} of $X$) and $\mathcal{T}$ is closed under unions, $W \in \mathcal{T}$.
\end{proof}

\begin{lemma}[13.3]
Let $\mathcal{B}$ and $\mathcal{B}'$ be bases for topologies $\mathcal{T}$ and $\mathcal{T}'$ on $X$. Then, the following are equivalent:

\begin{enumerate}
    \item $\mathcal{T}'$ is finer than $\mathcal{T}$
    \item for each $x \in X$ and each $B \in \mathcal{B}$ such that $x \in B$, there exists a basis element $B' \in \mathcal{B}'$ such that $x \in \mathcal{B}' \subset B$
\end{enumerate}
\end{lemma}

\begin{proof}
$(2) \implies (1)$. Given $U \in \mathcal{T}$, we want to show that $U \in \mathcal{T}'$. Let $
U \in X$. Since $\mathcal{B}$ generates $\mathcal{T}$, there exists $B \in \mathcal{B}$ such that $x \in B \subset U$. Hence, by assumption, there exists $B' \in \mathcal{B}'$ such that $x \in B' \subset B$ which implies that $x \in B' \subset U$ so $U \in \mathcal{T}'$.

\nline

$(1) \implies (2)$. Suppose we have $x \in X$ and $B \in \mathcal{B}$ with $x \in B$. By definition, $B \in \mathcal{T}$ and by assumption, $\mathcal{T} \subset \mathcal{T}'$. Hence, $B \in \mathcal{T}'$ so there exists some $B' \in \mathcal{B}'$ such that $x \in B' \subset B$ \com{(because $B$ is an open set of $\mathcal{T}'$)}.
\end{proof}

Three topologies on $\mathbb{R}$:

\begin{enumerate}
    \item The \textbf{standard topology} on $\mathbb{R}$ is generated by
    \[
    \mathcal{B} = \{ \text{the set of all open intervals} \; (a,b)  \{x : a < x < b\} \}.
    \]
    \item The \textbf{lower limit topology} on $\mathbb{R}$ ($\mathbb{R}_\ell$ is generated by
    \[
    \mathcal{B}' = \{ \text{all half open intervals} \; [a,b) = \{x : a \leq x < b\}.
    \]
    \item The \textbf{K-topology} on $\mathbb{R}$ ($\mathbb{R}_K$) is generated by
    \[
    \mathcal{B}'' = \{ \text{all intervals} \; (a,b) \; \text{and} \; (a,b) - K \; \text{where} \; K = \left\{ \frac{1}{n} : n \in \mathbb{Z} \right\} \}.
    \]
\end{enumerate}

\begin{lemma}[13.4]
The topologies $\mathbb{R}_\ell$ and $\mathbb{R}_K$ are strictly finer than $\mathbb{R}$ but are not comparable with each other.
\end{lemma}

A \textbf{subbasis} $\mathcal{S}$ for a topology on $X$ is a collection of subsets of $X$ whose union is equal to $X$. The topology generated by $\mathcal{S}$ is the collection $\mathcal{T}$ of all unions of finite intersections of elements of $\mathcal{S}$.

\nline

Sanity check: Is the topology generated by $\mathcal{S}$ really a topology? It suffices to check that that the collection $\mathcal{B}$ of all finite intersections of elements of $\mathcal{S}$ is a basis.

\begin{enumerate}
    \item Given some $x \in X$, we have $x \in S$ for some $S \in \mathcal{S}$ so $x \in B$ for some $B \in \mathcal{B}$.
    \item Let $B_1 = S_1 \cap \cdots \cap S_m$ and $B_2 = S_1' \cap \cdots \cap S_n'$ be two elements of $\mathcal{B}$. Their intersection $B_1 \cap B_2 = (S_1 \cap \cdots \cap S_m) \cap (S_1' \cap \cdots \cap S_n')$ is also a finite intersection of elements in $\mathcal{S}$.
\end{enumerate}

Remember that the second basis condition (``for all $B_1,B_2$ in $\mathcal{B}$ and $x \in B_1 \cap B_2$, there exists $B_3$ such that $x \in B_3 \subseteq B_1 \cap B_2$'') is true if the intersection of two basis elements is again a basis element. The case that involves a little more work is when two basis elements don't necessarily intersect to form another basis element (then you need to find a basis element \textit{inside} that intersection).

\subsection{The Order Topology}

Let $X$ be a set with a (total) order relation. Assume that $X$ has more than one element. Let $\mathcal{B}$ be the collection of all sets of the following types:
\begin{enumerate}
    \item all open intervals $(a,b) \in X$
    \item all intervals of the form $[a_0,b)$ where $a_0 = \min(X)$
    \item all intervals of the form $(a,b_0]$ where $b_0 = \max(X)$
\end{enumerate}
This is called the \textbf{order topology}.

\nline

Sanity check: Does this actually form a basis for a topology?

\begin{enumerate}
    \item Every element of $X$ lies in one of the above intervals (the smallest element is in every interval of type (2), the largest in type (3), and everything else in type (1)).
    \item The intersection of two intervals of types (1), (2), or (3), is again an interval of type (1), (2), or (3).
\end{enumerate}

Examples:

\begin{itemize}
    \item The standard topology on $\mathbb{R}$ \textit{is} the order topology.
    \item Lexicographic ordering gives us the order topology of $\mathbb{R} \times \mathbb{R}$.
    \item The long line topology $R = [0, 1) \times \omega_1$ with the dictionary order (where $\omega_1$ is smallest uncountable set).
\end{itemize}

\begin{center}
    \includegraphics[scale=0.5]{Julie/dictionaryorder.png}

    \caption{Dictionary order in $\mathbb{R} \times \mathbb{R}$}
\end{center}

\textbf{Open rays} of the form $(a, +\infty)$ and $(-\infty, a)$ form a subbasis for the order topology on $X$. Why? Open rays are first of all open in the order topology. Secondly, given an element $(a,b) \in X$, we can write $(a,b) = (-\infty, b) \cap (a, +\infty)$.

\subsection{Product Topology}

If $X, Y$ are topological spaces, the \textbf{product topology} on $X \times Y$ is the topology with basis
\[
\mathcal{B} = \{ U \times V : U \; \text{is open in} \; X; V \; \text{is open in} \; Y \}.
\]

Sanity check: Why does this give us a basis? Clearly, since $X \times Y$ is a basis element, every element of $X \times Y$ is contained in a basis element. Next, given two basis elements $U_1 \times V_1$ and $U_2 \times V_2$, we have
\[
(U_1 \times V_1) \cap (U_2 \times V_2) = (U_1 \cap U_2) \times (V_1 \cap V_2)
\]
which is again a basis element.

\begin{theorem}[15.1]
If $\mathcal{B}$ is a basis for a topology of $X$ and $\mathcal{C}$ is a basis for a topology of $Y$, then
\[
\mathcal{D} = \{ B \times C : B \in \mathcal{B}, C \in \mathcal{C} \}
\]
is a basis for a topology on $X \times Y$.
\end{theorem}

\begin{proof}
Given an open set $W \subset X \times Y$ and a point $x \times y \in W$, there exists a basis element $U \times V$ (where $U, V$ are open sets in $X, Y$) such that $x \times y \in U \times V \subset W$. Since $\mathcal{B}$ and $\mathcal{C}$ are bases for $B$ and $C$, we can choose basis elements $B, C$ such that $x \in B \subset U$ and $y \in C \subset V$. Then, $x \times y \in B \times C \subset W$ so by Lemma 13.2, $\mathcal{D}$ is a basis.
\end{proof}

\textit{Note:} The product topology (open rectangles) on $\mathbb{R}^2$ is the same as the metric topology (open discs) on $\mathbb{R}^2$.

\nline

Let
\[
\begin{align}
\pi_1: X \times Y &\longrightarrow X \qquad &\text{and} \qquad \pi_2: X \times Y &\longrightarrow Y \\
(x,y) &\longmapsto x \qquad &\text{} \qquad (x,y) &\longmapsto y.
\end{align}
\]
The maps $\pi_1, \pi_2$ are the (surjective) \textbf{projections} of $X \times Y$ onto $X$ and $Y$ respectively. In particular, if $U$ is open in $X$, then the set $\pi_1^{-1}(U)$ is the set $U \times Y$ which is open in $X \times Y$.

\begin{theorem}[15.2]
The collection
\[
\mathcal{S} = \{ \pi_1^{-1}(U) : U \; \text{is open in} \; X \} \cup \{ \pi_2^{-1}(V) : V \; \text{is open in} \; Y \}
\]
is a subbasis for the product topology on $X \times Y$.
\end{theorem}

\begin{proof}
Let $\mathcal{T}$ be the product topology on $X \times Y$ and let $\mathcal{T}'$ be the topology generated by $\mathcal{S}$. Since every element in $\pi_1^{-1}(U)$ and $\pi_2^{-1}(V)$ is open in $X \times Y$ \com{(why? $U$ and $V$ are open so $U \times Y$ and $X \times V$ are open)}, $\mathcal{S} \subseteq \mathcal{T}$. We also know that arbitrary unions of finite intersections of elements are in $\mathcal{T}$ so $\mathcal{T}' \subseteq \mathcal{T}$. Conversely, every basis element $U \times V$ for $\mathcal{T}$ is a finite intersection of elements in $\mathcal{S}$ because
\[
U \times V = \pi_1^{-1}(U) \cap \pi_2^{-1}(V)
\]
so $U \times V \in \mathcal{T}'$ so $\mathcal{T} \subseteq \mathcal{T}'$.
\end{proof}

\subsection{Subspace Topology}

Let $X$ be  a topological space with topology $\mathcal{T}$. If $Y \subseteq X$, the collection
\[
\mathcal{T}_Y = \{ Y \cap U : U \in \mathcal{T} \}
\]
is a topology on $Y$ called the \textbf{subspace topology}. A set is open in $\mathcal{T}_Y$ if it is the intersection of an open set of $X$ with all of $Y$.

Sanity check: Why is $\mathcal{T}_Y$ a topology?

\begin{enumerate}
    \item We see that $\varnothing = \varnothing \cap Y$ and $Y = X \cap Y$ so $\varnothing, Y \in \mathcal{T}_Y$.
    \item Arbitrary unions:
    \[
    \bigcup_{\alpha} (U_\alpha \cap Y) = (\bigcup_{\alpha} U_\alpha) \cap Y.
    \]
    \item Finite intersections: $(U_1 \cap Y) \cap \cdots \cap (U_n \cap Y) = (U_1 \cap \cdots \cap U_n) \cap Y$.
\end{enumerate}

\begin{lemma}[16.1]
If $\mathcal{B}$ is a basis for a topology of $X$, then
\[
\mathcal{B}_Y = \{ B \cap Y : B \in \mathcal{B} \}
\]
is a basis for the subspace topology on $Y$.
\end{lemma}

\begin{proof}
Suppose we have $U \in \mathcal{T}$ and $y \in U \cap Y$. We can choose $B \in \mathcal{B}$ such that $y \in B \subset U$ so $y \in B \cap Y \subset U \cap Y$ \com{(because if $y \in B$, then clearly $y \in B \cap Y$ because $y \in U \cap Y$ and $B \subset U$)}.
\end{proof}

\begin{lemma}[16.2]
("Transitivity") Let $Y$ be a subspace of $X$. If $U$ is open in $Y$ and $Y$ is open in $X$, then $U$ is open in $X$.
\end{lemma}

\begin{proof}
Since $U$ is open in $Y$, $U = Y \cap V$ for some open set $V \subseteq X$. Since $Y, V$ are both open in $X$, so is their intersection.
\end{proof}

\begin{theorem}[16.3]
If $A$ is a subspace of $X$ and $B$ a subspace of $Y$, then the product topology of the subspace topologies on $A \subseteq X$ and $B \subseteq Y$ is equal to the subspace topology on $A \times B$ of the product topology on $X \times Y$.
\end{theorem}

\begin{proof}
The set $X \times Y$ has basis elements of the form $U \times V$ with $U$ open in $X$ and $V$ open in $Y$. Hence, $(U \times V) \cap (A \times B)$ is a general basis element for the subspace topology on $A \times B$ \com{(here, $A \times B$ is the subset and $U \times V$ is the open set of $X \times Y$)}. Since
\[
(U \times V) \cap (A \times B) = (U \cap A) \times (V \cap B)
\]
where $U \cap A$ is open in the subspace topology on $A$ and $V \cap B$ is open in the subspace topology on $B$, we have a basis element for the product topology on $A \times B$ \com{(with $A, B$ as subspaces of $X, Y$)}. Since we have the same bases, the topologies are equal.
\end{proof}

Note: \textit{Sometimes} the subspace and order topologies on a set are the same. Oftentimes, they are \textit{not}. For example, let $X = \mathbb{R}$ and $Y = [0,1) \cup \{2\}$. In the subspace topology, $\{2\}$ is open because $\{2\} = (1.5, 2.5) \cap Y$. However, in the order topology, $\{2\}$ is \textit{not} open \com{(why? because every open subset must contain a basis element and $\{2\}$ by itself cannot be a basis element)}.

\nline

A subset $Y \subset X$ is \textbf{convex} in $X$ if, for each pair of points $a < b$ in $Y$, the entire interval $(a,b)$ of points in $X$ lies in $Y$.

\begin{theorem}[16.4]
Let $X$ be an ordered set equipped with the order topology. Let $Y$ be a subset of $X$ that is convex in $X$. Then the order topology on $Y$ is the same topology that $Y$ inherits as a subspace of $Y$.
\end{theorem}

\begin{proof}
Assume that $Y$ is convex and $a \in X$. We claim that $(-\infty, a) \cap Y$ is either empty, consists of all of $Y$, or is an open ray in $Y$. Why is this true?
\begin{itemize}
    \item If $a \in Y$, then $(-\infty, a) \cap Y = \{ x \in X : x \in Y \; \text{and} \; x < a \}$, but this is literally the definition of an open ray in $Y$.
    \item If $a \notin Y$, then $a$ is either a lower bound or an upper bound for $Y$. Why is this true? We know that $Y$ is convex so if $a \notin Y$ and $a$ is not an upper or lower bound for $Y$, then $a$ ``breaks'' the convexity of $Y$.

   % \begin{center}
        %\includegraphics[scale=0.05]{Julie/IMG_0054.jpg}
   % \end{center}

    If $a$ is a lower bound, then $(-\infty, a) \cap Y = \varnothing$. If $a$ is an upper bound, then $(-\infty, a) \cap Y = Y$.
\end{itemize}
Hence, $(-\infty, a) \cap Y$ is one of the three options and a similar proof shows this holds for $(a, +\infty) \cap Y$ as well. Since $\{ (-\infty, a) \cap Y, (a, +\infty) \cap Y \}$ is a subbasis for the subspace topology on $Y$ and each set is open in the order topology, the subspace topology is a subset of the order topology.

\nline

Conversely, any open ray of $Y$ is the intersection of an open ray of $X$ with all of $Y$ so open rays are open in the subspace topology on $Y$. Since the open rays of $Y$ are a subbasis for the order topology on $Y$, the order topology is a subset of the subspace topology.
\end{proof}

\subsection{Closed Sets and Limit Points}

A subset $A$ of a topological space $X$ is \textbf{closed} if $X - A$ is open.

\nline

Examples:
\begin{itemize}
    \item The set $\{ (x,y) : x \geq 0, y \geq 0 \}$ is closed because its complement
    \[
    \{ (-\infty, 0) \times \mathbb{R} \} \cup \{ \mathbb{R} \times (-\infty, 0) \}
    \]
    is open.
    \item In the finite complement topology, closed sets include $X$ itself and all finite subsets of $X$.
    \item In the discrete topology, every set is closed and open.
\end{itemize}

\begin{theorem}[17.1]
Let $X$ be a topological space; then
\begin{enumerate}
    \item $\varnothing$ and $X$ are closed
    \item Arbitrary intersections of closed sets are closed
    \item Finite unions of closed sets are closed
\end{enumerate}
\end{theorem}

\begin{proof}
\begin{enumerate}
    \item $\varnothing$ and $X$ are closed because they are the complements of $X$ and $\varnothing$.
    \item Given a collection of closed sets $\{A_\alpha\}$, then
    \[
    X - \bigcap_{\alpha} A_\alpha} = \bigcup_{\alpha} (X - A_\alpha).
    \]
    Since each $X - A_\alpha$ is open and arbitrary unions of open sets are open, the set on the right-hand side is open.
    \item Again, consider
    \[
    X - \bigcup_{i = 1}^n A_i = \bigcap_{i=1}^n (X - A_i).
    \]
    Since finite intersections are open, the set on the right-hand side is open.
\end{enumerate}
\end{proof}

\begin{theorem}[17.2]
Let $Y$ be a subspace of $X$. Then, a set $A$ is closed in $Y$ if and only if it equals the intersection of a closed set of $X$ with $Y$.
\end{theorem}

\begin{proof}
Suppose that $A = C \cap Y$ where $C$ is closed in $X$. Then $X - C$ is open in $X$ so $(X - C) \cap Y$ is open in $Y$. But $(X - C) \cap Y = X \cap Y - C \cap Y = Y - A$ \com{(because $Y \subseteq X$)} so $A$ is closed in $Y$.

\nline

Conversely, suppose that $A$ is closed in $Y$. Then $Y - A$ is open in $Y$ so $Y - A = U \cap Y$ where $U$ is an open subset of $X$. The set $X - U$ is closed in $X$ so because $A = Y \cap (X - U)$, $A$ is the intersection of a closed subset of $X$ with $Y$.
\end{proof}

\begin{theorem}[17.3]
Let $Y$ be a subspace of $X$. If $A$ is closed in $Y$ and $Y$ is closed in $X$, then $A$ is closed in $X$.
\end{theorem}

Given a subset $A$ of a topological space $X$, the \textbf{interior} of $A$ is the union of all open sets contained in $A$ and the \textbf{closure} of $A$ is the intersection of all closed sets containing $A$.

\[
\text{Int} \; A \subset A \subset \overline{A}
\]

If $A$ is open, $\text{Int} \; A = A$. If $A$ is closed, $\overline{A} = A$.

\begin{theorem}[17.4]
Let $Y$ be a subspace of $X$ and let $A$ be a subset of $Y$. Let $\overline{A}$ denote the closure of $A$ in $X$. Then, the closure of $A$ in $Y$ is $\overline{A} \cap Y$.
\end{theorem}

\begin{proof}
Let $B$ be the closure of $A$ in $Y$. The set $\overline{A}$ is closed in $X$ so $\overline{A} \cap Y$ is closed in $Y$ (Theorem 17.2). Since $A \subset \overline{A} \cap Y$ \com{(why? because $A \subset Y$ and $A \subset \overline{A}$)} and since, by definition, $B$ is the intersection of \textit{all} closed subsets of $Y$ containing $A$, we have $B \subseteq (\overline{A} \cap Y)$.

\nline

Conversely, we know that $B$ is closed in $Y$ (by definition). Hence, $B = C \cap Y$ for some closed set $C \subset X$. Then, $C$ is a closed set of $X$ containing $A$ \com{(why? because $B$ must contain $A$)}; since $\overline{A}$ is the intersection of \textit{all} closed sets containing $A$, we know that $\overline{A} \subset C$. Hence, $(\overline{A} \cap Y) \subset (C \cap Y) = B$.
\end{proof}

\begin{theorem}[17.5]
Let $A$ be a subset of the topological space $X$.
\begin{enumerate}
    \item $x \in \overline{A}$ if and only if every open set $U$ containing $x$ intersects $A$.
    \item If the topology of $X$ is given by a basis $\mathcal{B}$, $x \in \overline{A}$ if every basis element $B$ containing $x$ intersects $A$.
\end{enumerate}
\end{theorem}

\begin{proof}
We will prove the contrapositive statement: $x \in \overline{A}$ if and only if there exists a neighborhood $U$ of $x$ such that $U \cap A = \varnothing$.

\nline

If $x \notin \overline{A}$, let $U = X - \overline{A}$ (which is open because $\overline{A}$ is closed). We know that $x \in U$ and $U \cap \overline{A} = \varnothing$ which implies that $U \cap A = \varnothing$.

\nline

Conversely, suppose that there exists some neighborhood $U$ of $x$ such that $U \cap A = \varnothing$. Then, $X - U$ is closed and $A \subset X - U$ (because $U$ does not intersect $A$). Remember that the closure of $A$ is the intersection of all closed sets containing $A$. Hence, $\overline{A} \subseteq X - U$ which implies that $x \notin \overline{A}$.
\end{proof}

Examples
\begin{itemize}
    \item Let $X = \mathbb{R}$ and $A = (0,1]$. Then $\overline{A} = [0,1]$ because every neighborhood of $0$ intersects $A$.
    \item Consider $Y = (0,1]$ as a subspace of $\mathbb{R}$. The closure of the set $A = (0,\frac{1}{2})$ in $\mathbb{R}$ is $[0,\frac{1}{2}]$ and the closure in $Y$ is $[0, \frac{1}{2}] \cap Y = (0, \frac{1}{2}]$.
\end{itemize}

\nline

If $A$ is a subset of the topological space $X$ and $x \in X$, then $x$ is a \textbf{limit point} of $A$ if every neighborhood of $x$ intersects $A$ in some point other than $x$ itself.

\nline

Example: The only limit point of $B = \{ \frac{1}{n} : n \in \mathbb{Z}^+ \}$ is $0$ (because every other point either has a neighborhood that does not intersect $B$ or which only contains that point).

\begin{theorem}[17.6]
Let $A$ be a subset of a topological space $X$. Let $A'$ be the set of all limit points of $A$. Then $\overline{A} = A \cup A'$.
\end{theorem}

\begin{proof}
If $x \in A'$, then every neighborhood of $x$ intersects $A$ in some point. Thus, by Theorem 17.5, $x$ is an element of $\overline{A}$. Since we clearly have $A \subset \overline{A}$, we have $A \cup A' \subset \overline{A}$.

\nline

Conversely, suppose we have $x \in \overline{A}$. If $x \in A$, then obviously $x \in A \cup A'$. If $x \notin A$, then since every neighborhood of $x$ intersects $A$ in some point, each neighborhood must intersect $A$ in a point different from $x$. Hence, $x$ is a limit point of $A$ so $x \in A \cup A'$.
\end{proof}

\begin{corollary}[17.7]
A subset of a topological space is closed if and only if it contains all of its limit points.
\end{corollary}

\subsection{Hausdorff Spaces}

We say that a sequence of points $x_1, x_2, \dots$ in a space $X$ \textbf{converges} to a point $x \in X$ if, to each neighborhood $U$ of $x$, there exists a positive integer $N$ such that $x_n \in U$ for all $n \geq N$ (``all but finitely many points of the sequence are in the neighborhood'').

\nline

A topological space $X$ is a \textbf{Hausdorff space} if for each pair $x_1,x_2$ of distinct points of $X$, there exist disjoint neighborhoods $U_1, U_2$ of $x_1, x_2$.

\nline

Examples:

\begin{itemize}
    \item The discrete topology on any set is Hausdorff.
    \item $\mathbb{R}^n$ with the standard topology is Hausdorff.
    \item Every ordered set with its order topology in Hausdorff.
    \item Any set with the indiscrete topology is \textit{not} Hausdorff.
\end{itemize}

Essentially, ``Hausdorffness'' is a measure of how big open sets are in the topological space. A set is \textit{not} Hausdorff if the open sets are ``too big''.

\begin{theorem}[17.8]
Every finite set of points in a Hausdorff space $X$ is closed.
\end{theorem}

\begin{proof}
We really only need to show that the set consisting of a single point is closed. Suppose we have $\{x_0\}$. If $x \in X$ is distinct from $x_0$, then by Hausdorffness, we have two disjoint neighborhoods surrounding $x$ and $x_0$. Since the neighborhood surrounding $x$ does not intersect $\{x_0\}$, $x$ cannot be in the closure of $\{x_0\}$ (by Theorem 17.5). Since $x$ was an arbitrary point, that means that the only point that can be in the closure of $\{x_0\}$ is $x_0$ itself.
\end{proof}

The \textbf{$T_1$ axiom} is satisfied if every finite set of points in a topological space $X$ is closed.

\begin{theorem}[17.9]
Let $X$ be a topological space satisfying the $T_1$ axiom, and let $A$ be a subset of $X$. Then, $x \in X$ is a limit point of $A$ if and only if every neighborhood of $x$ contains infinitely many points of $A$.
\end{theorem}

\begin{proof}
If every neighborhood of $x$ intersects $A$ in infinitely many points, then it clearly intersects $A$ in at least one point distinct from $x$ so $x$ is a limit point of $A$.

\nline

Conversely, suppose that $x$ is a limit point of $A$ and there exists some neighborhood $U$ of $x$ that intersects $A$ in only finitely many points. Then $U$ also intersects $A - \{x\}$ in finitely many points; call these points $\{x_1,\dots,x_m\}$. The set $X - \{x_1,\dots,x_m\}$ is open in $X$ \com{(finite point sets are closed in a space with the $T_1$ axiom)}. Hence, $U \cap (X - \{x_1,\dots,x_m\})$ is a neighborhood of $x$ that does not intersect $A - \{x\}$ \com{(think: ``$U$ intersect with $X$ minus all the points that $U$ intersects $X$ with'')}. This contradicts $x$ being a limit point of $A$.
\end{proof}

\begin{theorem}[17.10]
If $X$ is a Hausdorff space, then a sequence of points in $X$ converges to at most one point of $X$.
\end{theorem}

\begin{proof}
Suppose that $\{x_n\}$ is a sequence of points that converges to $x$. If $y \neq x$, there exist disjoint neighborhoods $U, V$ of $x, y$. Since $U$ contains $x_n$ for all but finitely many $n$, $V$ cannot contain any of these $x_n$. Hence, $\{x_n\}$ cannot converge to $y$.
\end{proof}

\begin{theorem}[17.11]
The product of two Hausdorff spaces is also Hausdorff. A subspace of a Hausdorff space is also Hausdorff.
\end{theorem}

\subsection{Continuous Functions}

Let $X$ and $Y$ be topological spaces. A function $f: X \to Y$ is \textbf{continuous} if, for every open subset $V \subset Y$, the preimage $f^{-1}(V) \subset X$ is open.

\nline

\textit{Note:} Continuity is dependent upon the topologies defined on $X$ and $Y$!

\nline

If the topology on $Y$ is given by a basis $\mathcal{B}$, it is sufficient to show that the preimage of every basis element is open in $X$. Why? An arbitrary open set $U \in Y$ can be expressed as a union of basis elements
\[
U = \bigcup_{\alpha} B_\alpha,
\]
which means that
\[
f^{-1}(U) = \bigcup_{\alpha} f^{-1}(B_\alpha)
\]
so $f^{-1}(U)$ is open if the preimage of every basis element is open.

\nline

Similarly, if the topology on $Y$ is given by a subbasis, it is sufficient to show that the preimage of every subbasis element is open in $X$.

\begin{theorem}[18.1]
Let $X$ and $Y$ be topological spaces and suppose we have $f : X \to Y$. The following are equivalent:
\begin{enumerate}
    \item $f$ is continuous
    \item for every subset $A \subset X$, $f(\overline{A}) \subset \overline{f(A)}$
    \item for every closed set $B \subset Y$, the set $f^{-1}(B)$ is closed in $X$
    \item for each $x \in X$ and each neighborhood $V$ of $f(x)$, there exists a neighborhood $U$ of $x$ such that $f(U) \subset V$.
\end{enumerate}
\end{theorem}

\begin{proof}
We will show that $(1) \implies (2) \implies (3) \implies (1)$ and $(1) \implies (4)$ and $(4) \implies (1)$.

\nline

$(1) \implies (2)$. Suppose that $f$ is continuous. Let $A$ be a subset of $X$. We want to show that if $x \in f(\overline{A})$, then $x \in \overline{f(A)}$, i.e., if $x \in \overline{A}$, then $f(x) \in \overline{f(A)}$. Let $V$ be a neighborhood of $f(x)$. Then $f^{-1}(V)$ is an open set of $X$ containing $x$ (since $f$ is continuous); hence, $f^{-1}(V)$ must intersect $A$ at some point $y$ \com{(because $x \in \overline{A}$ so every open neighborhood of $x$ must intersect $A$ in some point)}. Then $V$ intersects $f(A)$ in some point $f(y)$ so $f(x) \in \overline{f(A)}$.

\nline

$(2) \implies (3)$. Let $B$ be closed in $Y$ and let $A = f^{-1}(B)$. We want to show that $A$ is closed in $X$; we will show that $A = \overline{A}$. We know that $f(A) = f(f^{-1}(B)) \subset B$ so if $x \in \overline{A}$, then
\[
f(x) \in f(\overline{A}) \subset \overline{f(A)} \subset \overline{B} = B
\]
so $x \in f^{-1}(B) = A$. Hence, $\overline{A} = A$.

\nline

$(3) \implies (1)$. Let $V$ be an open set of $Y$. Let $B = Y - V$ (so $B$ is closed). Then,
\[
f^{-1}(B) = f^{-1}(Y) - f^{-1}(V) = X - f^{-1}(V)
\]
so $f^{-1}(V)$ is open in $X$.

\nline

$(1) \implies (4)$. Let $x \in X$ and let $V$ be a neighborhood of $f(x)$. Then $U = f^{-1}(V)$ (which is open because $f$ is continuous) is a neighborhood of $x$ such that $f(U) \subseteq V$.

\nline

$(4) \implies (1)$. Let $V$ be an open set of $Y$ and suppose we have $x \in f^{-1}(V)$. Then $f(x) \in V$, so by hypothesis, there is an open neighborhood $U_x$ of $x$ such that $f(U_x) \subseteq V$. Then $U_x \subseteq f^{-1}(V)$. Hence, $f^{-1}(V)$ can be expressed as the union of open sets $U_x$ (because $x$ was just an arbitrary point picked from $f^{-1}(V)$) so it is open in $X$.
\end{proof}

Let $X$ and $Y$ be topological spaces and let $f : X \to Y$ be a bijection. If both the function $f$ and its inverse function $f^{-1}$ are continuous, then $f$ is a \textbf{homeomorphism} and $X$ and $Y$ are \textbf{homeomorphic}.

\begin{theorem}[18.2, rules for constructing continuous functions]
Let $X, Y, Z$ be topological spaces.
\begin{enumerate}
    \item (Constant function is continuous) If $f : X \to Y$ maps all of $X$ to a single point $y_0 \in Y$, then $f$ is continuous.
    \item (Inclusion map is continuous) If $A$ is a subspace of $X$, then function $j : A \to X$ is continuous.
    \item (Composition of continuous maps is continuous) If $f: X \to Y$ and $g: Y \to Z$ are continuous, then the map $g \circ f : X \to Z$ is continuous.
    \item (Continuous function with restricted domain is continuous) If $f: X \to Y$ is continuous and $A$ is a subspace of $X$, then $f|_{A}: A \to Y$ is continuous.
    \item (Continuous function with restricted or expanded range is continuous) Let $f: X \to Y$ be continuous. If $Z$ is a subspace of $Y$ containing the image set $f(X)$, then $g : X \to Z$ is continuous. If $Z$ is a space which contains $Y$ as a subspace, then $h: X \to Z$ is continuous.
    \item ("Local formulation of continuity") If $X = \bigcup_{\alpha} U_\alpha$ and $f|_{U_\alpha}$ is continuous for each $\alpha$, then $f: X \to Y$ is continuous.
\end{enumerate}
\end{theorem}

\begin{proof}
Let $X, Y, Z$ be topological spaces.
\begin{enumerate}
    \item Let $f(x) = y_0$ for all $x \in X$. Let $V$ be open in $Y$. The preimage $f^{-1}(V)$ is either all of $X$ (if $y_0 \in V$) or $\varnothing$ (if $y_0 \notin V$). In either case, it is open.
    \item If we have an open set $U \subseteq X$, then $j^{-1}(U) = U \cap A$ \com{(by definition of the inclusion map)}. But this is the definition of an open set in the subspace topology.
    \item If $U$ is open in $Z$, then $g^{-1}(U)$ is open in $Y$ and $f^{-1}(g^{-1}(U))$ is open in $X$. Note that $f^{-1}(g^{-1}(U)) = (g \circ f)^{-1}(U)$.
    \item Combine parts (3) and (4).
    \item Let $f: X \to Y$ be continuous. If $f(X) \subset Z \subset Y$, we want to show that $g : X \to Z$ is continuous. Let $B$ be open in $Z$. Then $B = Z \cap U$ for an open set $U \subseteq Y$ \com{(by definition of the subspace topology)}. Since $f(X) \subset Z$, $f^{-1}(U) = g^{-1}(B)$ and because $f^{-1}(U)$ is open, so is $g^{-1}(B)$ \com{(remember that $f$ is continuous!)}.
    \item Take some open set $V \subseteq Y$. Then
    \[
    \begin{align}
        f^{-1}(V) &= f^{-1}(V) \cap X \\
        &= f^{-1}(V) \cap (\bigcup_{\alpha} U_\alpha ) \\
        &= \bigcup_\alpha \left[f^{-1}(V) \cap U_\alpha \right] \\
        &= \bigcup_\alpha (f|_{U_\alpha})^{-1}(V).
    \end{align}
    \]
    Each set in the last expression is open so because unions of open sets are open, $f^{-1}(V)$ is open.
\end{enumerate}
\end{proof}

\begin{theorem}[18.3, the pasting lemma]
Let $X = A \cup B$ with $A,B$ closed in $X$. Let $f: A \to Y$ and $g: B \to Y$ be continuous. If $f(x) = g(x)$ for all $x \in A \cap B$, then $f$ and $g$ combine to give a continuous function $h: X \to Y$ defined by
\[
h(x) = \begin{cases}
f(x), \; \text{if} \; x \in A \\
g(x), \; \text{if} \; x \in B.
\end{cases}
\]
\end{theorem}

\begin{proof}
Let $C$ be a closed set of $Y$. Then $h^{-1}(C) = f^{-1}(C) \cup g^{-1}(C)$. The union of closed sets is closed.
\end{proof}

\subsection{The Metric Topology}

If $d$ is a metric on the set $X$, then the collection of all $\epsilon$-balls $B_\epsilon(x)$ for $x \in X$ and $\epsilon > 0$, is a basis for a topology on $X$, called the \textbf{metric topology induced by $d$}.

\nline

Sanity check: Why is this a basis?

\begin{enumerate}
    \item Clearly, $x \in B_\epsilon(x)$ for any $\epsilon > 0$.
    \item First, we show that if $y \in B_\epsilon(x)$, then there exists a basis element $B_\delta(y)$ centered at $y$ contained in $B_\epsilon(x)$. Let $\delta = \epsilon - d(x,y)$. Suppose we have $z \in B_\delta(y)$. Then $d(y,z) < \epsilon - d(x,y)$ so
    \[
    d(x,z) \leq d(x,y) + d(y,z) < d(x,y) + \epsilon - d(x,y) = \epsilon.
    \]
    Let $B_1, B_2$ be two basis elements and let $y \in B_1 \cap B_2$. We can choose $\delta_1,\delta_2$ such that $B_{\delta_1}(y) \subset B_1$ and $B_{\delta_2}(y) \subset B_2$. Let $\delta = \min(\delta_1,\delta_2)$. Then $B_\delta(y) \subset B_1 \cap B_2$.
\end{enumerate}

A set $U$ is open in the metric topology if and only if for each $y \in U$, there exists a $\delta > 0$ such that $B_\delta(y) \subset U$.

Example: The standard metric on $\mathbb{R}$ is defined by $d(x,y) = |x - y|$. This gives rise to the same topology as the order topology. \com{Why? Because if we take $(a,b)$ in the order topology, set $\epsilon = \frac{b-a}{2}$. Then $(\frac{a+b}{2} - \frac{b - a}{2}, \frac{a+b}{2} + \frac{b - a}{2})$ is the same interval.}

\nline

A topological space is \textbf{metrizable} if there exists a metric $d$ that induces the topology on the space.

\nline

\begin{theorem}[20.1]
Let $X$ be a metric space with metric $d$. Define $\overline{d}: X \times X \to \mathbb{R}$ by the function
\[
\overline{d}(x,y) = \min\{d(x,y), 1\}.
\]
Then $\overline{d}$ is a metric (the \textbf{standard bounded metric}) inducing the same topology as $d$.
\end{theorem}

\begin{proof}
The first two conditions are easily checked. For the triangle inequality, consider
\[
\overline{d}(x,z) \leq \overline{d}(x,y) + \overline{d}(y,z).
\]
Note that in any metric space, the collection of $\epsilon$-balls with $\epsilon < 1$ forms a basis for the metric topology.
\end{proof}

Examples:

\begin{itemize}
    \item The euclidean metric on $\mathbb{R}^n$:
    \[
    d(\textbf{x},\textbf{y}) = ||\textbf{x} - \textbf{y}|| = \sqrt{\sum_{i=1}^n (x_i - y_i)^2}.
    \]
    \item The ``taxicab'' metric:
    \[
    d(\textbf{x},\textbf{y}) = |x_1 - y_1| + \cdots + |x_n - y_n|.
    \]
    \item The square or $\ell_\infty$ metric:
    \[
    d(\textbf{x},\textbf{y}) = \max\{|x_1 - y_1|,\dots,|x_n - y_n|\}.
    \]
\end{itemize}

Note: In $\mathbb{R}$, the euclidean metric and $\ell_\infty$ metric give us the standard topology. In $\mathbb{R}^2$, the euclidean metric gives discs while the $\ell_\infty$ metric gives squares.

\begin{lemma}[20.2]
Let $d$ and $d'$ be two metrics on $X$; let $\mathcal{T}$ and $\mathcal{T}'$ be the topologies induced by these metrics. Then $\mathcal{T}'$ is finer than $\mathcal{T}$ if and only if for each $x \in X$ and each $\epsilon > 0$, there exists some $\delta > 0$ such that $B_\delta^{d'}(x) \subseteq B_\epsilon^d(x)$.
\end{lemma}

\begin{proof}
Suppose $\mathcal{T}'$ is finer than $\mathcal{T}$. Then, given $B_\epsilon^d(x)$, there exists a basis element $B'$ such that $x \in B' \subseteq B_\epsilon^d(x)$. We can clearly find a ball of radius $\delta$ centered at $x$ and contained in $B_\epsilon^d(x)$.

\nline

Conversely, suppose the right-hand condition holds. Then, given a basis element $B$ for $\mathcal{T}$ containing $x$, we can find a ball $B_\epsilon^d(x)$ contained in $B$. By hypothesis, we can choose some $\delta > 0$ such that $B_\delta^{d'}(x) \subseteq B_\epsilon^d(x)$. By Lemma 13.3, this implies that $\mathcal{T}'$ is finer than $\mathcal{T}$ \com{(i.e., there exists a basis element containing $x$ and contained in the other basis element)}.
\end{proof}

\begin{theorem}[20.3]
The topologies on $\mathbb{R}^n$ induced by the euclidean metric $d$ and the $\ell_\infty$ metric $\rho$ are the same as the product topology.
\end{theorem}

\begin{proof}
Let $\textbf{x},\textbf{y}$ be two points of $\mathbb{R}^n$. We can check that
\[
\rho(\textbf{x},\textbf{y}) \leq d(\textbf{x},\textbf{y}) \leq \sqrt{n} \rho(\textbf{x},\textbf{y}).
\]
From the first inequality, we see that $B_\epsilon^d(x) \subseteq B_\epsilon^\rho(x)$. The second inequality shows that $B_{\epsilon/\sqrt{n}}^\rho(x) \subseteq B_\epsilon^d(x)$ so by Lemma 20.2, these topologies are the same.

\nline

Now, we show that the topology induced by $\rho$ is the same as the product topology. Let $B = (a_1,b_1) \times (a_n,b_n)$ be a basis element of the product topology and let $\textbf{x} \in B$. For each $1 \leq i \leq n$, there exists $\epsilon_i$ such that $(x_i - \epsilon_i, x_i + \epsilon_i) \subseteq (a_i, b_i)$. Choose $\epsilon = \min\{\epsilon_1,\dots,\epsilon_n\}$. Then $B_\epsilon^\rho(\textbf{x}) \subseteq B$.

\nline

Conversely, let $B_\epsilon^\rho(\textbf{x})$ be a basis element for the $\rho$-topology. Given $\textbf{y} \in B_\epsilon^\rho(\textbf{x})$, we want to show that there exists a basis element $B$ for the product topology such that $\textbf{y} \in B \subseteq B_\epsilon^\rho(\textbf{x})$. But $B_\epsilon^\rho(\textbf{x}) = (x_1 - \epsilon, x_1 + \epsilon) \times \cdots \times (x_n - \epsilon, x_n + \epsilon)$ is itself a basis element for the product topology.
\end{proof}

Properties of the metric topology:

\begin{itemize}
    \item Subspaces: If $A$ is a subspace of the topological space $X$, where $d$ is a metric for $X$, then the restriction of $d$ to $A$ gives a metric topology on $A$.
    \item Hausdorffness: Every metric topology is Hausdorff. Why? If we have $x \neq y$, set $\epsilon = \frac{d(x,y)}{2}$. Suppose we have $z \in B_\epsilon(x) \cap B_\epsilon(y)$. Then
    \[
    d(x,y) \leq d(x,z) + d(z,y) < 2\epsilon = d(x,y),
    \]
    which is a contradiction.
\end{itemize}

\begin{theorem}
Let $X, Y$ be metrizable spaces with metrics $d_X$ and $d_Y$ respectively and suppose we have $f : X \to Y$. Then, $f$ is continuous if and only if given $x \in X$ and $\epsilon > 0$, there exists $\delta > 0$ such that $d_X(x,y) < \delta$ implies $d_Y(f(x),f(y)) < \epsilon$.
\end{theorem}

\begin{proof}
Suppose that $f$ is continuous. Given $x$, $\epsilon$, consider $f^{-1}(B_\epsilon(f(x)))$ which is open in $X$ (because $f$ is continuous) and contains $x$ \com{(not necessarily centered at $x$)}. By the fact that balls are open, there exists some $\delta$-ball centered at $x$ and contained in the preimage of $B_\epsilon(f(x))$. If $y$ is in this $\delta$-ball \com{(i.e., $d_X(x,y) < \delta$)}, then $f(y)$ is in the $\epsilon$-ball centered at $f(x)$ in $Y$.

\nline

Conversely, suppose that the $\epsilon-\delta$ condition is satisfied. Let $V$ be open in $Y$. We want to show that $f^{-1}(V)$ is open in $X$. Suppose we have $x \in f^{-1}(V)$. Since this means that $f(x) \in V$, there exists an $\epsilon$-ball centered at $f(x)$ and contained in $V$ \com{(because neighborhoods are open)}. By the $\epsilon-\delta$ condition, there exists a $\delta$-ball centered at $x$ which is contained in the $\epsilon$-ball centered at $f(x)$. This means that the $\delta$-ball around $x$ is an open set contained in $f^{-1}(V)$. Since $x$ was arbitrarily chosen, this means that $f^{-1}(V)$ is open.
\end{proof}

\begin{lemma}[21.2]
Let $X$ be a topological space; let $A \subseteq X$. If there is a sequence of points of $A$ converging to $x$, then $x \in \overline{A}$. The converse holds \textit{if} $X$ is metrizable.
\end{lemma}

\begin{proof}
Suppose that $x_n \to x$ where each $x_n \in A$. Then every neighborhood of $x$ contains a point of $A$ so $x \in \overline{A}$. On the other hand, suppose that $X$ is metrizable and $x \in \overline{A}$. We want to construct a sequence converging to $x$. For each $n \in \mathbb{Z}^+$, take the neighborhood $B_{1/n}(x)$ and choose $x_n$ to be its point of intersection with $A$ \com{(every neighborhood of $x$ must intersect $A$)}. Then, any open set $U$ containing $x$ contains the ball $B_\epsilon(x)$. If we choose $N$ so that $\frac{1}{N} < \epsilon$, then $U$ contains $x_i$ for $i \geq N$.
\end{proof}

\begin{theorem}[21.3]
Let $f : X \to Y$. If the function $f$ is continuous, then for every convergent sequence $x_n \to x$ in $X$, the sequence $f(x_n)$ converges to $f(x)$. The converse holds \textit{if} $X$ is metrizable.
\end{theorem}

\begin{proof}
Suppose that $f$ is continuous. Given $x_n \to x$, we want to show that $f(x_n) \to f(x)$. Let $V$ be a neighborhood of $f(x)$. Then $f^{-1}(V)$ is a neighborhood of $x$ \com{(because $f$ is continuous)}. Hence, there exists some $N$ such that $x_n \in f^{-1}(V)$ for all $n \geq N$, which implies that $f(x_n) \in V$ for all $n \geq N$.

\nline

Conversely, suppose that $X$ is metrizable and the convergent sequence condition is satisfied. Let $A \subseteq X$. We want to show that $f(\overline{A}) \subseteq \overline{f(A)}$ \com{(remember that this comes from Theorem 18.1(2))}. If $x \in \overline{A}$, then there exists a sequence of points $x_n$ in $A$ converging to $x$ (by Lemma 21.2). By assumption, $f(x_n)$ converges to $f(x)$. Since $f(x_n) \in f(A)$, the preceding lemma implies that $f(x) \in \overline{f(A)}$. Hence, since $x \in \overline{A}$ implies $f(x) \in f(\overline{A})$, we have $f(\overline{A}) \subseteq \overline{f(A)}$.
\end{proof}

\subsection{The Quotient Topology}

Let $X, Y$ be topological spaces and let $p : X \to Y$ be a surjective map. The map $p$ is a \textbf{quotient map} if: a subset $U \subseteq Y$ is open if and only if $q^{-1}(U)$ is open in $X$.

\nline

Example: Let $S^1 = \{ z \in \mathbb{C} : |z| = 1\}$ (i.e., the unit circle in the complex plane). Define the map $q: \mathbb{R} \to S^1$ by $q(t) = e^{it}$. We claim that $q$ is a quotient map.
\begin{itemize}
    \item Why is $q$ continuous? We know that $q(t) = (\cos(t), \sin(t)) = \cos(t) + i\sin(t)$ where each of these functions is analytically continuous (note that a pair of continuous functions defines a continuous function in the product topology).
    \item Why is $q$ an open map? We want to show that the image of an open interval is open. If the ``length'' of $(a,b)$ is greater than $2\pi$, then $q[(a,b)]$ ``covers'' the whole circle (i.e., is the entirety of $S^1$). If not, then there exists a line $\ell$ which passes through $q(a)$ and $q(b)$. Let $U \subseteq \mathbb{R}^2$ be the subspace in the complement of $\ell$ containing the image set. Then $U \cap S^1 = q(a,b)$ is open.
\end{itemize}

Let $X$ be a topological space. Let $\sim$ be an equivalence relation on $X$. We define a \textbf{quotient set} $X / \sim$ to be the set of equivalent classes of $\sim$ in $X$. Let $q : X \to X/\sim$ be the surjective map which sends an element $x \in X$ to its equivalence class $[x] \in X / \sim$. We define the \textbf{quotient topology} on $X / \sim$ by saying that $U \subseteq X / \sim$ is open if and only if $q^{-1}(U)$ is open \com{(note that this preimage is a collection of elements in the equivalence classes that belong to $U$)}.

\begin{theorem}
Let $q : X \to Y$ be a quotient map. Let $f : X \to Z$ be a continuous map with the property that for all $x, x' \in q^{-1}(y)$, $f(x) = f(x')$ \com{(``two points in $x$ going to the same point in $Y$ means that they go to the same point in $Z$'')}. Then there exists a unique continuous function $\overline{f} : Y \to Z$ such that $\overline{f} \circ q = f$.
\end{theorem}

\begin{proof}
We need to show three things: (1) existence of $\overline{f}$, (2) uniqueness of $\overline{f}$, and (3) continuity of $\overline{f}$.
\begin{enumerate}
    \item Suppose we have $y \in Y$. Pick $x \in q^{-1}(y) \neq \varnothing$ (which is possible because $q$ is surjective). Define $\overline{f}(y) = f(x)$. Why does this work? If $x' \in q^{-1}(y)$, then $f(x') = f(x)$ by assuption so $\overline{f}$ is indeed well-defined.
    \item Suppose that $g : Y \to Z$ such that $g \circ q = f$. We want to show that $g = \overline{f}$. If $y \in Y$ and $x \in g^{-1}(y)$, then $g(y) = g(q(x)) = f(x) = \overline{f}(y)$.
    \item We want to show that if $U \subseteq Z$ is open, then $\overline{f}^{-1}(U)$ is open in $Y$. This is equivalent to saying that $q^{-1}(\overline{f}^{-1}(U))$ is open in $X$ because $q$ is a quotient map \com{(i.e., $q^{-1}(\overline{f}^{-1}(U))$ is open if and only if $\overline{f}^{-1}(U)$ is open)}. But we know that $q^{-1}(\overline{f}^{-1}(U)) = f^{-1}(U)$ (by definition of $\overline{f}$) and $f^{-1}(U)$ is open because $f$ is continuous.
\end{enumerate}
\end{proof}

\section{Connectedness and Compactness}

\subsection{Connected Spaces}

Let $X$ be a topological space. A \textbf{separation} of $X$ is a pair $A, B$ of disjoint nonempty open subsets of $X$ such that $X = A \cup B$. The space $X$ is \textbf{connected} if there does not exist a separation of $X$.

\nline

\textit{Alternative definition:} A space $X$ is connected if the only clopen sets are $X$ and $\varnothing$.

\begin{lemma}[23.1]
If $Y$ is a subspace of $X$, a separation of $Y$ is a pair of disjoint nonempty sets $A$ and $B$ whose union is $Y$, neither of which contains a limit point of the other. The space $Y$ is connected if there exists no separation of $Y$.
\end{lemma}

\begin{proof}
Suppose that $A, B$ form a separation of $Y$. Then $A$ is clopen in $Y$ (because $A = Y - B$ and $B$ is open as well). We know that $A = \overline{A} \cap Y$ (where $\overline{A}$ denotes the closure of $A$ in $X$) which means that $\overline{A} \cap B = \varnothing$. Similarly, we have $\overline{B} \cap A = \varnothing$.

\nline

Conversely, suppose that $A,B$ are disjoint nonempty subsets whose union is $Y$, neither of which contain a limit point of the other. This means that $\overline{A} \cap B = \varnothing$ and $A \cap \overline{B} = \varnothing$ so $\overline{A} \cap Y = A$ and $\overline{B} \cap Y = B$. This implies that $A, B$ are both closed and open.
\end{proof}

\begin{lemma}[23.2]
If the sets $C, D$ form a separation of $X$, and if $Y$ is a connected subspace of $X$, then $Y$ lies either entirely within $C$ or within $D$.
\end{lemma}

\begin{proof}
Since $C,D$ are open in $X$, the sets $C \cap Y$ and $D \cap Y$ are open in $Y$ (by definition of subspace topology). We see that $C \cap Y$ and $D \cap Y$ are disjoint (because $C,D$ are disjoint) and their union is all of $Y$. If they were both nonempty, then they would constitute a separation of $Y$, which is a contradiction.
\end{proof}

\begin{theorem}[23.3]
The union of a collection of connected subspaces of $X$ that have a point in common is connected.
\end{theorem}

\begin{proof}
The intuition here is that the collection of connected subspaces need to ``overlap'' in order to form a connected union. Let $\{A_\alpha\}$ be a collection of connected subspaces and let $p \in \bigcap A_\alpha$. We want to show that $Y = \bigcup A_\alpha$ is connected. Suppose that $Y = C \cup D$ is a separation of $Y$. Then either $p \in C$ or $p \in D$; suppose that $p \in C$. Since each $A_\alpha$ is connected, it must lie entirely in $C$ (by the previous lemma) because it contains $p \in C$. Hence, $D$ is empty which is a contradiction.
\end{proof}

\begin{theorem}[23.5]
The image of a connected space under a continuous map is connected.
\end{theorem}

\begin{proof}
Let $f : X \to Y$ be a continuous map and let $Z = f(X)$. Suppose that $Z = A \cup B$, where $A,B$ form a separation of $Z$. Then, $f^{-1}(A) \cup f^{-1}(B)$ is a separation of $X$ ($f^{-1}(A)$ and $f^{-1}(B)$ are open because $f$ is continuous).
\end{proof}

\begin{theorem}
The set of real numbers $\mathbb{R}$ is connected.
\end{theorem}

\begin{proof}
Suppose $\mathbb{R}$ is disconnected. Write $\mathbb{R} = A \cup B$ where $A,B$ are open, disjoint, and nonempty. Pick $a \in A$ and $b \in B$. Without loss of generality, assume $a < b$. Consider $[a,b]$ as a topological space with the subspace topology (as a subset of $\mathbb{R}$). There exists a separation $A_0 = A \cap [a,b]$ and $B_0 = B \cap [a,b]$ \com{(each of these sets are open because of the definition of the subspace topology, disjoint because $A$ and $B$ are disjoint, and nonempty because $a \in A$ and $b \in B$)}. By force, $A_0$ has an upper bound; for example, $b$ works as an upper bound for $A_0$. This implies that $A_0$ has a least upper bound; let $c = \sup A_0$. Clearly, $c \in [a,b]$ because $b$ is an upper bound so $c \leq b$. Since $[a,b] = A_0 \cup B_0$, either $c \in A_0$ or $c \in B_0$. If $c \in A_0$, then $c$ is the largest element of $A_0$. However, remember that $A_0$ is open so it must contain a neighborhood of radius $\epsilon > 0$ around $c$, which of course means that $c$ cannot be the largest element. Now suppose $c \in B_0$. Since $c$ is the least upper bound of $A_0$, we know that $c \in \overline{A_0}$. This means that every neighborhood of $c$ intersects $A_0$. Since $B_0$ is open, there exists a neighborhood around $c$ contained in $B_0$. But this neighborhood must intersect $A_0$ so $B_0$ intersects $A_0$ nontrivially.
\end{proof}

Examples of homeomorphisms:

\begin{itemize}
    \item If $(a,b)$ and $(c,d)$ are open nonempty intervals in the standard topology on $\mathbb{R}$, then they are homeomorphic under
    \[
    f(x) = \frac{d-c}{b-a} (x - a) c.
    \]
    \item Consequently, $(a,b] \cong (c,d]$, $[a,b) \cong [c,d)$, and $[a,b] \cong [c,d]$.
    \item The interval $\left-\frac{\pi}{2}, \frac{\pi}{2}\right)$ is homeomorphic to $\mathbb{R}$ under $f(x) = \tan(x)$.
\end{itemize}

\begin{theorem}[23.6]
If $X, Y$ are connected spaces, then $X \times Y$ is also connected.
\end{theorem}

\begin{proof}
Pick $x_0 \in X$. For all $y \in Y$, define $U_{y_0} = (\{x_0\} \times Y) \cup (X \times \{y_0\}$. We claim that $U_{y_0}$ is connected because it is the union of two connected spaces where $\{x_0\} \times Y$ is homeomorphic to $Y$ and $X \times \{y_0\}$ is homeomorphic to $X$. The intersection of these two spaces is nonempty (because both contain the point $x \times y$) so by Theorem 23.3, $U_{y_0}$ is connected. What is $\bigcap_{y \in Y} U_y$? Since $\{x_0\} \times Y$ has fixed $x$ and we note that $\bigcap_{y \in Y} X \times \{y\} = \varnothing$, we see that $\bigcap_{y \in Y} U_y = \{x\} \times Y$ which is clearly nonempty. Hence, by Theorem 23.3, the set $\bigcup_{y \in Y} U_y$ is connected. What is this last set? We can imagine that the lines corresponding to $x_0$ and $y_0$ form a set of ``cross-hairs'' across $X \times Y$. As $y$ varies, it ``sweeps'' out the entirety of $X \times Y$ so hence, $X \times Y$ is connected.
\end{proof}

\begin{lemma}
Let $A$ be a connected subspace of a topological space $X$ ($X$ need not be connected). Let $B$ be a set such that $A \subseteq B \subseteq \overline{A}$. Then $B$ is also connected.
\end{lemma}

\begin{proof}
The intuition here is that $\overline{A}$ is the ``skin'' on the topological space $A$. Suppose that $B$ is \textit{not} connected. Let $B = C \cup D$ where $C,D$ are open, disjoint, and nonempty. Since $A \subseteq B$, we must $A \subseteq C$ or $A \subseteq D$ because otherwise, $A$ would be disconnected. Without loss of generality, suppose $A \subseteq C$. Then $D \subseteq B \subseteq \overline{A}$ but $D \cap A = \varnothing$. Hence, $D$ is an open set which intersects $\overline{A}$ nontrivially (in fact, $D \subseteq \overline{A}$) but does not intersect $A$, which is a contradiction.
\end{proof}

\begin{theorem}
If $X$ is homeomorphic to $Y$, then $X$ is connected if and only if $Y$ is connected.
\end{theorem}

\begin{proof}
A quotient space of a connected space is connected but recall that a homeomorphism, by definition, \textit{is} a quotient map.
\end{proof}

Let $x, y \in X$. Then a \textbf{path} from $x \to y$ is a continuous function $\gamma: [0,1] \to X$ such that $\gamma(0) = x$ and $\gamma(1) = y$. $X$ is \textbf{path-connected} if for all $x, y \in X$, there exists a path from $x \to y$.

\begin{lemma}
If $X$ is path-connected, then it is connected.
\end{lemma}

\begin{proof}
Suppose $X$ is path-connected but \textit{not} connected. Then $X = A \cup B$ is a separation of $X$ into two nonempty, open, and disjoint sets. Since $A,B$ are nonempty, pick $a \in A$ and $b \in B$. Pick any path $\gamma: [0,1] \to X$. Since $\gamma([0,1])$ is the image of a connected set under a continuous function, it is also connected. This implies that either $\gamma([0,1]) \subseteq A$ or $\gamma([0,1]) \subseteq B$. Since this is true for \textit{every} path in $X$, there exists no path from $a \to b$ which contradicts the assumption that $X$ is path-connected.
\end{proof}

\textit{Note.} Connected does \textit{not} imply path-connected.

\nline

Example: The topologist's sine curve.

\begin{itemize}
    \item Suppose we have $f: (0, 1] \to \mathbb{R}^2$ where $f(x) = (x, \sin\left(\frac{1}{x}\right))$. Let $S = \; \text{im} \; f$.
    \item Theorem: $\overline{S}$ is connected but not path-connected. Proof: $S$ is connected because it is the image of a connected set $(0,1]$ under a continuous map. By Lemma 3.1.3, this means that $\overline{S}$ is also connected. We want to show that $\overline{S} = S \cup (\{0\} \times [-1,1])$. (Take this for granted...) Now, we want to prove that $\overline{S}$ is not path-connected. Suppose $\overline{S}$ is path-connected. Let $\gamma$ be a path with $\gamma(0) = 0 \times 0$ and $\gamma(1) = 1 \times \sin(1)$. Then there exists some maximum $t$ such that $\gamma(t) \in \{0\} \times [-1,1]$ \com{(the intuition: the path eventually must leave the edge line that sits along the $y$-axis and traverse towards the point $1 \times \sin(1)$)}. Define the function $\rho(s) = \gamma((1-t)s+t)$ \com{(re-parametrizing $\gamma$ so that the $0$-point is moved to $t$ and re-scaling the rest so that it ends at the point $1$)}. The rest of the proof proceeds by showing that there exists a sequence of points $s_n \to 0$ such that $\rho(s_n) = (-1)^n$. But this implies that $\lim_{n \to \infty} \rho(s_n)$ does not exist which contradicts the continuity of $\rho$.
\end{itemize}

\begin{theorem}[intermediate value theorem]
Let $X$ be a connected space; let $Y$ be an ordered set with the order topology. Let $a,b \in X$ and let $f$ be a continuous function from $X \to Y$. If $f(a) < r < f(b)$, then there exists some $c \in X$ such that $r = f(x)$.
\end{theorem}

\begin{proof}
Suppose this is not the case. Consider $U = (-\infty, r)$ and $V = (r, \infty)$. Then we claim that $f^{-1}(U) \cup f^{-1}(V)$ form a separation of $X$. Why is this the case? Clearly, $f^{-1}(U) \cup f^{-1}(V) = X$ because $U \cup V = Y$. The sets are disjoint because $U$ and $V$ are disjoint \com{($f$ is well-defined!)}. Both sets are open because $f$ is continuous. Finally, both sets are nonempty because $U$ and $V$ are nonempty. We have arrived at a contradiction because we assumed that $X$ is connected.
\end{proof}

\subsection{Components and Local Connectedness}

Given $X$, define an equivalence relation on $X$ by setting $x \sim y$ if there exists a connected subset $A \subseteq X$ containing $x$ and $y$.

\nline

Sanity check: Why is this an equivalence relation?

\begin{enumerate}
    \item Clearly, the singleton set $\{x\}$ is connected so $x \sim x$.
    \item Symmetry comes by definition.
    \item If we have $x,y \in A$ and $y,z \in B$ where $A, B$ connected, then $A,B$ both contain $y$ so $A \cup B$ is a connected set containing $x,z$.
\end{enumerate}

\textit{Note.} If $X$ is connected, then all the elements of $X$ are in the same equivalence class.

\nline

The set of \textbf{components} of $X$ is the set of equivalence classes $X / \sim$.

\begin{theorem}[25.1]
The components of $X$ are connected, disjoint subspaces of $X$ such that each connected subspace of $X$ intersects only one of them.
\end{theorem}

\begin{proof}
Clearly, since the components of $X$ are equivalence classes, they are disjoint and their union is all of $X$. If $A$ intersects two components $C_1, C_2$, then we pick $x_1 \in C_1$ and $x_2 \in C_2$ such that $x_1 \sim x_2$. This is a contradiction.

\nline

Pick a point $x_0 \in X_{x_0}$. Then for all $x \in X_{x_0}$, there exists a connected subspace $A_x \subseteq X$ with $x, x_0 \in A_x$. Hence, $\bigcup_{x \in X_{x_0}} A_x = X_{x_0}$. But we have just written $X_{x_0}$ as a union of connected spaces with the point $x_0$ in common so each component must be connected.
\end{proof}

\textit{Note.} The components need not be open. For example, the components of $\mathbb{Q}$ are the singleton sets.

\nline

Define another equivalence relation on $X$ by $x \sim y$ if there exists a path in $X$ from $x \to y$. These equivalence classes are the \textbf{path components} of $X$.

\nline

A space $X$ is \textbf{locally connected} at some point $x \in X$ if, for every open neighborhood $U \ni x$, there exists a connected open neighborhood $V \ni x$ such that $V \subseteq U$. \com{(The intuition here is that we can have ``arbitrarily small'' open connected neighborhoods around each point.)}

\nline

Example: $\mathbb{R}$ is locally connected because if we pick $x \in \mathbb{R}$ and we take any open set $U \ni x$, then $U$ contains a basis element $(a,b) \ni x$ which must be connected.

\begin{theorem}[25.3]
A space $X$ is locally connected if and only if for every open set $U \subseteq X$, all the components of $U$ are open.
\end{theorem}

\begin{proof}
Suppose that $X$ is locally connected. Let $U \subseteq X$ be open and let $C \subseteq U$ be a component of $U$. If $x \in C$, then we can choose a connected neighborhood $V$ of $x$ such that $V \subseteq U$ \com{(by definition of local connectedness)}. Since $V$ is connected, it must lie entirely within the component $C$. Hence, $C$ is open because we can write it as the union of $V_x$ for each $x \in C$ where $V_x \subseteq C$.

\nline

Conversely, suppose that the components of open sets in $X$ are open. Given a point $x \in X$ and a neighborhood $U \ni x$, let $C$ be the component of $U$ containing $x$. Since $C$ is by definition connected and by hypothesis open, $C \subseteq U$ implies that $C$ works as our open, connected neighborhood containing $x$.
\end{proof}

\begin{theorem}[25.5]
If $X$ is a topological space, each path component of $X$ lies in a component of $X$. If $X$ is locally path-connected, then the path components and the components of $X$ are the same.
\end{theorem}

\begin{proof}
Let $C$ be a component of $X$ and let $x \in C$. Let $P$ be the path component of $X$ containing $x$. Since path components are path-connected and thus connected, $P \subseteq C$. We now want to show that if $X$ is locally path-connected, $P = C$. Suppose that this is not the case. Then let $Q$ denote the union of all path components of $X$ different from $P$ and which intersect $C$. We know that $C = P \cup Q$. Since $X$ is locally path-connected, each path component of $X$ is open in $X$. Thus, $P$ (which is a path component) and $Q$ (which is a union of path components) are both open in $X$ so they form a separation of $C$. This is a contradiction.
\end{proof}

\subsection{Compact Spaces}

A space $X$ is \textbf{compact} if every open cover has a finite subcover.

\begin{lemma}[26.1]
Let $Y$ be a subspace of $X$. Then $Y$ is compact if and only if every covering of $Y$ by sets open in $X$ contains a finite subcollection covering $Y$.
\end{lemma}

\begin{proof}
(Note: This result says that compactness in a general space is equivalent to compactness in a subspace.) Suppose $Y \subseteq \bigcup_{\alpha \in A} U_\alpha$. Define $V_\alpha = U_\alpha \cap Y$. Each $V_\alpha$ is open in $Y$ by definition of the subspace topology. Clearly, $Y = \bigcup_{\alpha \in A} V_{\alpha}$. But if we throw away $U_{\alpha_i}$ in $X$, this is equivalent to throwing away the corresponding $V_{\alpha_i}$ in $Y$. Hence, $Y$ has a finite subcover. (The proof of the converse follows similarly.)
\end{proof}

\begin{theorem}[26.2]
Every closed subspace of a compact space is compact.
\end{theorem}

\begin{proof}
Suppose $X$ is compact and $Y \subseteq X$ is closed. Then $X - Y$ is open. Let $\bigcup_{\alpha \in A} U_\alpha$ be an open covering of $Y$ by sets $U_\alpha$ open in $X$. Adjoin $X - Y$ to this open cover. Find a finite subcover. Throw out $X - Y$ if necessary.
\end{proof}

\begin{theorem}[26.3]
Every compact subspace of a Hausdorff space is closed.
\end{theorem}

\begin{proof}
Let $Y$ be a compact subspace of a Hausdorff space $X$. We want to show that $X - Y$ is open by showing that for each $x \in X - Y$, there exists an open set $U$ such that $x \in U \subseteq X - Y$ \com{(i.e., allowing us to express $X - Y$ as a union of open sets)}. Since $X$ is Hausdorff, for every $x \in X - Y$ and $y \in Y$, we can pick $U_y \ni x$ and $V_y \ni y$ such that $U_y \cap V_y = \varnothing$. Fix $x \in X - Y$ and pick $V_y$ for each $y \in Y$ so that $U_y \cap V_y = \varnothing$. Since $Y$ is compact and $Y = \bigcup_{y \in Y} V_y$, we see that $Y \subseteq V_{y_1} \cap \cdots \cap V_{y_n}$ for some $n$. Then $U = U_{y_1} \cap \cdots \cap U_{y_n}$ is an open set containing $x$ disjoint from $Y$. \com{(Why? Because it is an intersection of neighborhoods containing $x$.)}
\end{proof}

\begin{theorem}[26.5]
The image of a compact set under a continuous map is compact.
\end{theorem}

\begin{proof}
Let $X$ be a compact set and $f: X \to Y$ be a continuous function. Let $U = \bigcup_{\alpha \in A} U_\alpha$ be a covering of $f(X)$ by sets open in $Y$. Then the collection of sets $f^{-1}(U)$ is an open cover of $X$ (because $f$ is continuous). We can find a finite subcover $\{f^{-1}(U_{\alpha_1}), \dots, f^{-1}(U_{\alpha_n})\}$. Correspondingly, the sets $\{U_{\alpha_1},\dots,U_{\alpha_n}\}$ cover $f(X)$.
\end{proof}

\begin{theorem}[26.6]
Let $f : X \to Y$ be a bijective continuous map. If $X$ is a compact space and $Y$ is a Hausdorff space, then $f$ is a homeomorphism.
\end{theorem}

\begin{proof}
Note that we have all the ``pieces'' required for a homeomorphism except for the requirement that $f$ be a closed (or open) map. We will show that the image of closed sets in $X$ under $f$ are closed. If $A$ is closed in $X$, then $A$ is compact in $X$ by Theorem 26.2. By Theorem 26.5, $f(A)$ is compact. By Theorem 26.3, $f(A)$ is closed.
\end{proof}

\begin{theorem}[26.7]
The product of finitely many compact spaces is compact.
\end{theorem}

\begin{proof}
See the Tychonoff theorem (actually, arbitrary products of compact spaces are compact).
\end{proof}

\subsection{Compact Subspaces of the Real Line}

\begin{theorem}[27.1]
Let $X$ be a totally ordered set with the least upper bound property. In the order topology, each closed interval in $X$ is compact.
\end{theorem}

\begin{proof}
Suppose $a < b$. Let $\mathcal{A}$ be an open cover of $[a,b]$ by sets open in the subspace topology. We want to show that there exists a finite subcollection of $\mathcal{A}$ that covers $[a,b]$.
\begin{enumerate}
    \item Step 1: \textit{Show that, for a point $x \in [a,b)$, there exists $y > x$ such that $[x,y]$ can be covered by at most two elements of $\mathcal{A}$.} If $x$ has an immediate successor, then let $y$ be this successor. Hence, $[x,y]$ has two elements so can be covered by at most two elements of $\mathcal{A}$. If $x$ does \textit{not} have an immediate successor, choose $A \in \mathcal{A}$ containing $x$. Since $x \neq b$, we know that $[x,c) \subset A$ for some $c \in [a,b]$. Pick $y \in [x,c)$. Then $[x,y]$ is covered by a single element $A \in \mathcal{A}$.
    \item Step 2: Let $C$ be the set of all points $y > a$ in $[a,b]$ such that $[a,y]$ can be covered by finitely many sets of $\mathcal{A}$. When $x = a$, we can apply Step 1 to show that there exists $y$ such that $[a,y]$ can be covered by finitely many elements of $\mathcal{A}$. Hence, $C$ is nonempty. Furthermore, $C$ is clearly bounded above by $b$. Let $c$ be the least upper bound of $C$.
    \item Step 3: \textit{Show that $c \in C$.} Choose an element $A \in \mathcal{A}$ containing $c$. Since $A$ is open, it must contain an interval $(d,c]$ for some $d \in [a,b]$. Suppose $c \notin C$. Then, there exists some $z \in (d,c)$ (because otherwise, $d$ would be a least upper bound for $C$). Since $z \in C$, the interval $[a,z]$ can be covered by finitely many elements of $\mathcal{A}$. Now, the interval $[z,c]$ is completely contained in $A$ so $[a,c] = [a,z] \cap [z,c]$ can be covered by finitely many elements of $\mathcal{A}$. Hence, $c \in C$.
    \item Step 4: \textit{Show that $c = b$.} Suppose that $c < b$. Then, there exists $y > c$ such that $[c,y]$ can be covered by finitely many elements of $\mathcal{A}$. But by Step 3, $[a,c]$ can be covered by finitely many elements of $\mathcal{A}$ so $[a,y] = [a,c] \cap [c,y]$ has the same property. But we have found $y > c$ such that $y \in C$ which contradicts the fact that $c$ is the least upper bound of $C$. Hence, $c = b$ so $[a,b]$ can be covered by finitely many elements of $\mathcal{A}$.
\end{enumerate}
\end{proof}

\begin{corollary}[27.2]
Every closed interval of $\mathbb{R}$ is compact.
\end{corollary}

\begin{theorem}[27.3, Heine-Borel]
A subspace $A \subseteq \mathbb{R}^n$ is compact if and only if it is closed and bounded in either the euclidean metric $d$ or the square ($\ell_\infty$) metric $\rho$.
\end{theorem}

\begin{proof}
Note first that
\[
\rho(x,y) \leq d(x,y) \leq \sqrt{n} \rho (x,y)
\]
so boundedness in both metrics is equivalent. We will use $\rho$.

\nline

Suppose that $A$ is compact. Then by Theorem 26.3, because $\mathbb{R}^n$ is Hausdorff, $A$ is closed. Consider the set of open ``balls'' of radius $m$ around $0$ for all $m \in \mathbb{N}$. This is an open cover of $\mathbb{R}^n$ and hence of $A$. We can pick a finite subcover $\{B_{m_1},\dots,B_{m_k}\}$. Let $M = \max(m_1,\dots,m_k)$. For any two points $x,y \in A$, we know that $\rho(x,y) \leq d(x,0) + d(9,y) < 2M$.

\nline

Conversely, suppose that $A$ is closed and bounded. Since $A$ is bounded, there exists $N \in \mathbb{R}$ such that $\rho(x,y) < N$ for all $x,y \in A$. Pick $\beta \in A$. Then set $\rho(0, \beta) = b$. For any $x \in A$,
\[
\rho(0, x) \leq \rho(\beta, 0) + \rho(\beta, x) < b + N = P.
\]
Then $A \subseteq [-P, P]^n$. Since closed intervals in $\mathbb{R}$ are compact, $[-P,P]^n$ is a finite product of compact sets and is hence compact. Finally, $A$ is a closed subset of a compact space so we conclude that $A$ is compact.
\end{proof}

\begin{theorem}[27.4, Extreme Value Theorem]
Let $f : X \to Y$ be a continuous function with $Y$ an ordered set equipped with the order topology. If $X$ is compact, then there exist points $c, d \in X$ such that $f(c) \leq f(x) \leq f(d)$ for each $x \in X$.
\end{theorem}

\begin{proof}
Let $A = f(X) \subseteq Y$. There are two cases to consider.
\begin{enumerate}
    \item If $A$ has a largest element $d_0 \in A$, then $d_0 = f(b)$ for some $b \in X$. For all $c \in X$, we have $f(c) \leq f(b)$.
    \item If $A$ does \textit{not} have a largest element, then cover $A$ by open sets $(-\infty, d)$ for all $d \in A$. Since $f$ is continuous, the preimage of this open cover is an open cover of $X$. Find a finite subcover of the preimage and let $d_i$ be the largest element such that $(-\infty, d_i)$ is in the subcover. Now we have $d_i \in A$ but which is not in our finite subcover of $A$. This is a contradiction.
\end{enumerate}
\end{proof}

Recall that in a metric space, a function $f: X \to Y$ is continuous at $x \in X$ if for all $\epsilon > 0$, there exists $\delta$ such that if $d(x,y) < \delta$, $d(f(x),f(y)) < \epsilon$.

\nline

A function $f$ is \textbf{uniformly continuous} if for all $\epsilon > 0$, there exists $\delta > 0$ such that if $d(x,y) < \delta$, $d(f(x),f(y)) < \epsilon$, \textit{for all} $x \in X$.

\nline

Let $X$ be a metric space and let $A \subseteq X$. For each $x \in X$, define the \textbf{distance from $x$ to $A$} by
\[
d(x,A) = \inf \{ d(x,a) : a \in A \}.
\]

\begin{lemma}[27.5, Lebesgue number lemma]
Let $\mathcal{A}$ be an open covering of the metric space $X$. If $X$ is compact, there exists $\delta > 0$ such that for each subset of $X$ with diameter $< \delta$, there exists an element of $\mathcal{A}$ which completely contains it.
\end{lemma}

\begin{proof}
Let $\mathcal{A}$ be an open cover of $X$. If $X \in \mathcal{A}$, then any subset of $X$ can be covered by the element $X \in \mathcal{A}$, regardless of $\delta$.

\nline

Suppose $X \notin \mathcal{A}$. Since $X$ is compact, pick $\{A_1,\dots,A_n\} \in \mathcal{A}$ which cover $X$. Set $C_i = X - A_i$. Define $f : X \to \mathbb{R}$ by
\[
f(x) = \frac{1}{n} \sum_{i=1}^n d(x,C_i).
\]
\com{(Note: For any point $x \in X$, $f(x)$ gives the average distance from $x$ to the sets $C_i$ for all $i$.)} We want to show that $f(x) > 0$ for all $x$. Given $x \in X$, choose $A_i$ such that $x \in A_i$ \com{(we have a \textit{cover} for $X$)}. Since $A_i$ is open, we can pick $\epsilon$ so that $x \in B_\epsilon(x) \subseteq A_i$. Then $d(x,C_i) \geq \epsilon$ for all $C_i$ so $f(x) \geq \epsilon/n$ which is strictly positive.

\nline

Since $f$ is a continuous function on a compact set, it must achieve a minimum value $\delta$ \com{(Extreme Value Theorem)}. We want to show that this $\delta$ is the Lebesgue number. Let $B \subseteq X$ such that $\text{diam} \, B < \delta$. Pick a point $x_0 \in B$. Clearly, $B \subseteq B_\delta(x_0)$ \com{(because $\delta$ was the \textit{diameter} of $B$ and the \textit{radius} of our open ball)}. We know that
\[
\delta \leq f(x_0) \leq d(x_0,C_m),
\]
where $d(x_0,C_m)$ is the maximum distance \com{(remember that $\delta$ is the \textit{minimum} value)}. Then the $\delta$-ball around $x_0$ is in $A_m = X - C_m$.
\end{proof}

\begin{theorem}[27.6]
Let $f: X \to Y$ be a continuous map from a compact metric space $(X, d_X)$ to a metric space $(Y, d_Y)$. Then $f$ is uniformly continuous.
\end{theorem}

\begin{proof}
Given $\epsilon > 0$, take the open covering of $Y$ by $B_{\epsilon/2}(y)$ for each $y \in Y$. Then the set of $f^{-1}(B_{\epsilon/2}(y))$ is an open cover of $X$ (because $f$ is continuous \textit{and a function}). Choose $\delta$ to be the Lebesgue number for $\mathcal{A}$. Then if $x_1, x_2 \in X$ such that $d_X(x_1,x_2) < \delta$, the set $\{x_1,x_2\}$ clearly has diameter less than $\delta$ so its image $f(\{x_1,x_2\})$ lies in some open ball $B_{\epsilon/2}(y)$ for some $y \in Y$. Then $d_Y(f(x_1),f(x_2)) < \epsilon$ so $f$ is uniformly continuous.
\end{proof}

\subsection{Limit Point Compactness}

A space $X$ is \textbf{limit point compact} if every infinite subset of $X$ has a limit point  in $X$.

\begin{theorem}[28.1]
Compactness implies limit point compactness (but not necessarily the other way!)
\end{theorem}

\begin{proof}
Let $X$ be a compact space. We want to show that if a subset $A \subseteq X$ has no limit point in $X$, then $A$ must be finite. If $A$ contains no limit points, then $A$ vacuously contains all its limit points so $A$ is closed. For each $a \in A$, we can pick $U_a \subseteq X$ such that $U_a \cap A = \{a\}$ \com{(by definition of a limit point)}. Clearly, $X = (X - A) \cup U_{a_i} \cup \cdots$ and since $X$ is compact, there exists a finite subcover. Since $X - A$ does not intersect $A$ and the intersection of each $U_{a_i}$ with $A$ consists of one point, $A$ must be finite.
\end{proof}

\begin{theorem}[28.2]
Let $X$ be a metrizable space. The following are equivalent:

\begin{enumerate}
    \item $X$ is compact.
    \item $X$ is limit point compact.
    \item $X$ is sequentially compact.
\end{enumerate}
\end{theorem}

\subsection{Local Compactness}

A space $X$ is \textbf{locally compact} if there exists some compact subspace $C$ of $X$ that contains a neighborhood of $x$.
\[
x \in U_x \subseteq C \subseteq X
\]
\begin{itemize}
    \item If $X$ is compact, $X$ must be locally compact (let $C = X$).
    \item $\mathbb{R}$ is locally compact
    \[
    x \in (x - \epsilon, x + \epsilon) \subset [x - \epsilon, x + \epsilon] \subset \mathbb{R}
    \]
    but not compact.
\end{itemize}

\begin{theorem}[29.1]
Let $X$ be a space. Then $X$ is locally compact and Hausdorff if and only if there exists a space $Y$ satisfying the following conditions:
\begin{enumerate}
    \item $X$ is a subspace of $Y$.
    \item The set $Y - X$ consists of a single point.
    \item $Y$ is a compact Hausdorff space.
\end{enumerate}
If $Y$ and $Y'$ are two spaces satisfying these conditions, then $Y$ is homeomorphic to $Y'$ by a unique homeomorphism which is the identity on $X$.
\end{theorem}

\begin{proof}
To prove this, we need to show several things: (1) if $X$ is locally compact and Hausdorff, such a space $Y$ exists and is a topological space, (2) $Y$ is compact and Hausdorff, (3) if $X$ has a one-point compactification $Y$ satisfying the three conditions, then $X$ is locally compact and Hausdorff, and (4) if $Y$ and $Y'$ satisfy the conditions, then there exists a unique homeomorphism from $Y \to Y'$ which acts as the identity on $X$.

\begin{enumerate}
    \item Suppose $X$ is locally compact and Hausdorff. Define the space $X^* = X \cup \{\infty\}$. Clearly, if $Y = X^*$, then $Y - X = \{\infty\}$. We need to verify that $Y$ is a topological space, and is both compact and Hausdorff.

    \nline

    Define a topology on $X^*$ to consist of two types of open sets:

    \begin{enumerate}
        \item If $U \not\ni \infty$, $U$ is open in $X^*$ if and only if $U$ is open in $X$.
        \item If $U \ni \infty$, $U$ is open in $X^*$ if and only if $X^* - U \subseteq X$ is a compact subspace of $X$.
    \end{enumerate}

    Why is this a topology on $X^*$?

    \begin{enumerate}
        \item $X^* = X^* - \varnothing$ and $\varnothing$ is clearly compact. Note that $\varnothing \subseteq X$ is open and is thus open in $X^*$.
        \item If $U_1, U_2 \subseteq X$, then $U_1 \cap U_2$ is open in $X$ by the topology on $X$. If $U_1, U_2$ both contain $\infty$, then $U_1 = X^* - C_1$ and $U_2 = X^* - C_2$ with $C_1, C_2$ compact. Since finite unions of compact sets are also compact, $U_1 \cap U_2 = X^* - (C_1 \cup C_2)$ is open. Finally, if $U_2$ contains $\infty$ but $U_1$ does not, then $U_1 \cap U_2 = U_1 \cap (X^* - C_2) = U_1 \cap (X - C_2) = U_1 - C_2$. Since $C_2$ is a compact subset of a Hausdorff space ($X$), $C_2$ is closed which implies that $U_1 \cap U_2$ is open.
        \item Let $U_\alpha$ be an arbitrary collection of open sets. If each $U_\alpha \subset X$, then the conclusion is immediate. If each $U_\alpha$ contains $\infty$, then $U_\alpha = X^* - C_\alpha$, where $C_\alpha$ is compact. Hence,
        \[
        \bigcup U_\alpha = X^* - \bigcap C_\alpha.
        \]
        But since each $C_\alpha$ is a compact subset of a Hausdorff space, each $C_\alpha$ is closed and since arbitrary intersections of closed sets are also closed, $\bigcap C_\alpha$ is closed so $\bigcup U_\alpha$ is open. Finally, if we have a combination of open sets of type (a) and type (b), then for each set of type (a), we know $U_\alpha = X - C_\alpha$ where $C_\alpha$ is closed. For each set of type (b), we know $U_\alpha = X^* - D_\alpha$ where $D_\alpha$ is compact. Hence, the union over all these sets is $\bigcup U_\alpha = X^* - \left(\bigcap C_\alpha \cap \bigcap D_\alpha \right)$ which is open because we have the complement of an arbitrary intersection of closed sets.
    \end{enumerate}
    Why is $X$ a subspace of $X^*$? We want to show that given any open set $U \subseteq X^*$, its intersection with $X$ is open in $X$. If $U \subseteq X$, the conclusion is immediate. If $U$ contains $\infty$, then $U = X^* - C$ so $U \cap X = X \cap (X^* - C) = X - C$ where $C$ is compact and thus closed.
    \item We now want to show that $X^*$ is compact and Hausdorff. If $\{U_\alpha\}$ is an open cover of $X^*$, we know that $\infty \in U_{\alpha_0}$ for some $\alpha_0$. We also know that $U_{\alpha_0}$ must be the complement of some compact set in $X^*$. Hence, $U_{\alpha_0} = X^* - C$. Since $C$ is compact, there exists a finite subcollection of $\{U_\alpha\}$ which covers it. But $U_{\alpha_0}$ covers the rest of $X^*$.

    \nline

    To show that $X^*$ is Hausdorff, we want to show that given $x,y \in X^*$, there exist $U \ni x$, $V \ni y$ such that $U \cap V = \varnothing$. Suppose that $x,y \in X$. Since $X$ is Hausdorff, we can find $U, V$ which satisfy the above properties and which are open in $X$. They are open in $X^*$ as well because of the subspace topology. If say $y = \infty$, then by local compactness of $X$, there exists a compact set $C \subseteq X$ such that $x \in U \subseteq C \subseteq X$ with $U$ open. Let $V = X^* - C$ which contains $\infty$ because $C \subseteq X$. Then $U \cap V = U \cap (X^* - C) = \varnothing$ because $U \subseteq C$.

    \item Now, we want to show that if we have a space $Y$ which satisfies the three conditions, this implies that $X$ must be locally compact and Hausdorff. Clearly, $X$ is Hausdorff because $X$ is a subspace of $Y$ which is Hausdorff by the third condition. Why is $X$ locally compact? Pick $x \in X$. Clearly, $x \neq \infty$ because $\infty \in Y - X$. Since $Y$ is Hausdorff, there exist open sets $U, V \subseteq Y$ such that $x \in U$, $\infty \in V$ and $U \cap V = \varnothing$. We know that $Y - V$ is closed (because $V$ is open) and since $Y$ is compact and closed subspaces of compact spaces are themselves compact, $Y - V$ is compact. Furthermore, $Y - V \subseteq X$ because $V$ contains $\infty$ and we are taking its complement. Hence, we have
    \[
    x \in U \subseteq Y - V \subseteq X.
    \]

    \item Finally, we want to show the uniqueness condition. We first show that a homeomorphism exists between $Y$ and $Y'$ which is the identity on $X$. If $Y, Y'$ satisfy the three conditions, define $f: Y \to Y'$ such that
    \[
    f(x) = \begin{cases}
    x \quad \text{if $x \in X$} \\
    \infty' \quad \text{if $x = \infty$}.
    \end{cases}
    \]
    Why is $f$ continuous? Let $U \subseteq Y'$ be open. If $U \subseteq X$, then $f^{-1}(U) = U$ (because $f$ is the identity on $X$) so we know that $U$ is open in $X$. We want to show that $U$ is open in $Y$. We claim that $X$ is open in $Y$. Since $Y$ is Hausdorff, singletons are closed so $\{\infty\}$ is closed. Hence, since $X = Y - \{\infty\}$, $X$ is open so $U \subseteq X \subseteq Y$ is open.

    \nline

    On the other hand, if $U \not\subseteq X$, then $\infty' \in U$. Since $Y' - U$ is closed and is a subset of a compact space $Y'$, $Y' - U$ is also compact. But note that $Y' - U \subseteq X$ because $\infty' \in U$. Let $C = Y' - U$ ($C$ is compact subspace of $X$). Since $X$ is also a subspace of $Y$, $C$ is a compact subspace of $Y$. Finally, because $Y$ is Hausdorff, $C$ is closed in $Y$ so $U$ is open in $Y$.

    \nline

    Note that bijectivity is obvious and $f$ is clearly a unique homeomorphism (every point in $X$ must be mapped to itself and there is only one other point).
\end{enumerate}
\end{proof}

If $Y$ is a compact Hausdorff space and $X$ is a proper subspace of $Y$ whose closure is equal to $Y$, then $Y$ is a \textbf{compactification} of $X$. If $Y - X$ is a single point, then $Y$ is the \textbf{one-point compactification} of $X$.

\begin{itemize}
    \item The one-point compactification of the real line $\mathbb{R}$ is homeomorphic to the unit circle.
    \item The one-point compactification of $\mathbb{R}^2$ is homeomorphic to the sphere $S^2$.
\end{itemize}

\section{Countability and Separation Axioms}

\subsection{The Countability Axioms}

A topological space $X$ is \textbf{second-countable} if there exists a countable basis.

\nline

Example: $\mathbb{R}^n$ with the standard topology. Why? Take $\{(a_1,b_1) \times \cdots \times (a_n, b_n)\}$ to be a countable basis with $a_i, b_i \in \mathbb{Q}$.

\begin{theorem}[30.2]
Subspaces and products of second-countable spaces are second-countable.
\end{theorem}

\begin{theorem}[30.3]
Suppose that $X$ is second-countable. Then
\begin{itemize}
    \item Every open cover of $X$ has a countable subcover (\textbf{Lindel\"of}).
    \item There exists a countable dense subset $D \subseteq X$ (\textbf{separable}).
\end{itemize}
\end{theorem}

\begin{proof}
Let $B = \{B_n\}$ be a countable basis for $X$. We will first prove that $X$ is Lindel\"of. Pick an open cover $\{U_\alpha\}$ of $X$. Define $A_n$ such that
\[
A_n = \begin{cases}
U_\alpha &\quad \text{if there exists} \; B_n \subseteq U_\alpha \\
\varnothing &\quad \text{otherwise}.
\end{cases}
\]
(We are only adding in the open sets in $\{U_\alpha\}$ which contain basis elements. If there exist multiple open sets $U_\alpha$ which contain $B_n$, then we just pick an arbitrary one.) Then $A_n$ is countable because it is indexed by a subset of a countable set, open, and we claim that it covers $X$. Why is this the case? Suppose we have some $x \in X$. Then $x \in U_\alpha$ for some $\alpha$. Since $B$ is a basis for $X$, there exists some $B_n$ such that $x \in B_n \subseteq U_\alpha$ (by definition of open set w.r.t a basis). Hence, $x \in B_n \subseteq A_n$.

\nline

Next, we prove that $X$ is separable. Pick an element $d_n \in B_n$ for each $n$. Let $D = \{d_n\}$. We claim that $D$ is dense in $X$, i.e., $\overline{D} = X$. Let $U \ni x$ be an open set. Pick a basis element $B_n \in B$ such that $x \in B_n \subseteq U$. If we intersect $B_n$ with $D$, we see that $d_n \in B_n \cap D$ so since $B_n \subseteq U$, $U \cap D \neq \varnothing$. Hence, every open set containing $x$ intersects $D$ nontrivially so $x$ is a limit point of $D$. Since $x$ was arbitrarily picked, this implies that \textit{every} point of $X$ is a point or limit point of $X$.
\end{proof}

\subsection{The Separation Axioms}

A space is:

\begin{itemize}
    \item $T_1$ if points are closed.
    \item $T_2$ if Hausdorff.
    \item $T_3$ (or \textbf{regular}) if points are closed and if, given a point $x$ and a closed set $B$, there exist open sets separating them.
    \item $T_3$ (or \textbf{normal}) if points are closed and if any two closed sets can be separated by two open sets.
\end{itemize}

\begin{center}
    \includegraphics[scale=0.5]{Julie/screenshot.png}
\end{center}

\begin{lemma}[31.1]
Let $X$ be a topological space where points are closed ($T_1$).
\begin{enumerate}
    \item $X$ is regular if and only if given $x \in X$ and $U \ni x$, there exists $V \ni x$ such that $\overline{V} \subseteq U$.
    \item $X$ is normal if and only if given a closed set $A$ and an open set $U \supseteq A$, there exists $V \supseteq A$ such that $\overline{V} \subseteq U$.
\end{enumerate}
\end{lemma}

\begin{proof}
We will prove $(1)$. The proof of $(2)$ is similar (replacing ``point'' with ``closed set''). Suppose that $X$ is regular. Pick a point $x \in X$ and an open set $U \ni x$. Set $B = X - U$ ($B$ is closed). Note that $B \cap \{x\} = \varnothing$. By regularity, there exist open sets $V, W$ such that $x \in V$ and $B \subseteq W$ with $V \cap W = \varnothing$. Note that $B \cap \overline{V} = \varnothing$ because if we have some $y \in B \subseteq W$, then $W$ is a neighborhood of $y$ which does not intersect $V$ so $y$ cannot be a limit point of $V$. This implies that $\overline{V} \subseteq U$.

\nline

On the other hand, suppose we have a point $x$ and a closed set $B$ such that $x \notin B$. Let $U = X - B$. By hypothesis, since $x \in U$ with $U$ open, there exists $V \ni x$ such that $\overline{V} \subseteq U$. But then the sets $V$ and $X - \overline{V}$ are open sets that separate $x$ and $B$ which implies that $X$ is regular.
\end{proof}

\begin{corollary}[31.2]
\textcolor{white}{txt}
\begin{enumerate}
    \item Subspaces of Hausdorff spaces are Hausdorff. Products of Hausdorff spaces are Hausdorff.
    \item Subspaces of regular spaces are regular. Products of regular spaces are regular.
\end{enumerate}
\end{corollary}

\subsection{Normal Spaces}

\begin{theorem}[32.1]
If $X$ is regular and second-countable, then $X$ is normal.
\end{theorem}

\begin{proof}
Let $X$ be a regular space with a countable basis $\mathcal{W}$. Take disjoint closed subsets $A, B \subseteq X$. For each point $x \in A$, we can pick an open set $U_x \ni x$ which does not intersect $B$. By Lemma 31.1, we can pick an open set $V_x \ni x$ such that $\overline{V_x} \subseteq U_x$. By the definition of a basis, we can pick $W_n \in \mathcal{W}$ such that $x \in W_n \subseteq V_x$. Since we can pick a basis element for each $x \in A$ in this way, we have a countable basis for $A$ where the closure of each basis element does not intersect $B$. Let's call this new basis $\{U_n\}$.

\nline

Pick a countable basis for $B$ in a similar manner: $\{V_n\}$. Set $U = \bigcup U_n$ and $V = \bigcup V_n$. We would hope that these two open sets would work as a separation of our original closed sets $A, B$ but we cannot guarantee that they are disjoint. Hence, set
\[
U_n' = U_n - \bigcup_{i=1}^n \overline{V_i} \quad \text{and} \quad V_n' = V_n - \bigcup_{i=1}^n \overline{U_i},
\]
Clearly, each $U_n'$ and $V_n'$ are open because we have an open set minus the finite union of closed sets. We also know that $\{U_n'\}$ covers $A$ because if we pick some $x \in A$, then $x \in U_n$ for some $n$ (by definition of basis) and $x \notin \overline{V_i}$ for any $i$ because $\overline{V_i} \cap A = \varnothing$ by construction. In the same way, we see that $\{V_n'\}$ covers all of $B$.

\nline

All we have left to do is prove that $U' = \bigcup U_n'$ and $V' = \bigcup V_n'$ are disjoint. Suppose not. Then there exists some $z \in U' \cap V'$ so $z \in U_n'$ and $z \in V_m'$. Without loss of generality, assume that $m \leq n$. Since $z \in U_n' = U_n - \bigcup_{i=1}^n \overline{V_i}$, this implies that $z \notin V_i$ for all $i \leq n$ so $z \notin V_m$. This means that $z \notin V_m'$ because $V_m' \subseteq V_m$, which is a contradiction.
\end{proof}

\begin{theorem}[32.2]
Every metrizable space is normal.
\end{theorem}

\begin{proof}
Let $X$ be a metrizable space with a metric $d$. Let $A, B$ be disjoint closed subsets of $X$. For each $a \in A$, pick an open ball of radius $\epsilon_a > 0$ such that $B_{\epsilon_a}(a) \cap B = \varnothing$. Why does such a ball exist? If not, then $a \in \overline{B}$ but since $B$ is closed, $\overline{B} = B$ so $a \in B$ which is a contradiction. Pick an open ball around each $b \in B$ in a similar fashion.

\nline

Set
\[
U = \bigcup_{a \in A} B_{\epsilon_a/2}(a) \quad \text{and} \quad V = \bigcup_{b \in B} B_{\epsilon_b/2}(b).
\]
We know that $U, V$ cover $A, B$ respectively because we have open balls around each point in $A$ and $B$. We also claim that $U$ and $V$ are disjoint. Why? If not, then we have some $z \in U \cap V$ which implies that $z \in B_{\epsilon_a/2}(a)$ for some $a$ and $z \in B_{\epsilon_b/2}(b)$ for some $b$. This implies that
\[
\begin{align}
    d(a, b) &\leq d(a, z) + d(z, b) \\
    &< \frac{\epsilon_a}{2} + \frac{\epsilon_b}{2} \\
    &\leq \max \{\epsilon_a, \epsilon_b\}.
\end{align}
\]
Without loss of generality, suppose that $\epsilon_a \leq \epsilon_b$. This means that $a \in B_{\epsilon_b/2}(b)$ which is a contradiction.
\end{proof}

\begin{theorem}[32.3]
Every compact Hausdorff space is normal.
\end{theorem}

\begin{theorem}[32.4]
Every well-ordered set $X$ is normal in the order topology.
\end{theorem}

\begin{proof}
Two preliminaries: (1) $X$ is well-ordered if every nonempty subset has a minimum element, and (2) $(x, y]$ is an open set in a well-ordered set with the order topology because $(x, y] = (x, y')$ where $y'$ is the immediate successor of $y$.

\nline

Let $X$ be a well-ordered set equipped with the order topology. Let $A, B$ be disjoint closed sets in $X$. Let $a_0$ be the smallest element of $X$ which exists because $X$ is a well-ordered set. We have two cases to consider: (1) the case where neither $A$ nor $B$ contain $a_0$, and (2) the case where either $A$ or $B$ (\textit{not} both) contain $a_0$.

\nline

First suppose that $a_0 \notin A$ and $a_0 \notin B$. For each $a \in A$, there exists a basis element containing $a$ which is disjoint from $B$ (why? $B$ is closed so its complement is open; since $a \in B^C$, there exists an open set around $a$ completely contained in $B^C$). Let's take the open interval $(x_a, a]$ to be this basis element where $x_a$ is *some* element smaller than $a$ (note that this uses the fact that $a \neq a_0$ for all $a$). Similarly, pick basis elements $(y_b, b]$ for each $b \in B$ such that $(y_b, b] \cap A = \varnothing$. Set
\[
U = \bigcup_{a \in A} (x_a, a] \quad \text{and} \quad V = \bigcup_{b \in B} (y_b, b].
\]
Clearly, $U$ and $V$ cover $A$ and $B$ respectively. We claim that $U$ and $V$ are disjoint. If not, we have some $z \in U \cap V$. This implies that $z \in (x_a, a]$ and $z \in (y_b, b]$ for some $a, b$. Without loss of generality, suppose that $a < b$. Then, since $x_a < z \leq a$ and $y_b < z \leq b$, we see that $y_b < z \leq a < b$ so $a \in (y_b, b]$ which is a contradiction.

\nline

Now suppose that $a_0 \in A$. We know that $\{a_0\}$ is open because $\{a_0\} = [a_0, a_1)$ where $a_1$ is the immediate successor of $a_0$. This implies that $A' = A - \{a_0\}$ is closed. By the previous argument, there exist open sets $U, V$ separating $A'$ and $B$. Then $U \cup \{a_0\}$ and $V$ are disjoint sets separating $A$ and $B$.
\end{proof}

\subsection{The Urysohn Lemma}

If $A$ and $B$ are two subsets of a topological space $X$ and if there exists a continuous function $f: X \to [0,1]$ such that $f(A) = \{0\}$ and $f(B) = \{1\}$, then $A$ and $B$ can be \textbf{separated by a continuous function}.

\nline

A space $X$ is \textbf{completely regular} if one-point sets are closed in $X$ and for each point $x_0 \in X$ and each closed set $A$ not containing $x_0$, there exists a continuous function separating $x_0$ and $A$.

\begin{theorem}[33.1, the Urysohn Lemma]
Let $X$ be a normal space. Let $A, B$ be disjoint, closed subsets of $X$. Let $[a,b]$ be a closed interval in the real line. Then there exists a continuous map $f: X \to [a,b]$ such that $f(x) = a$ for each $x \in A$ and $f(x) = b$ for each $x \in B$.
\end{theorem}

\begin{proof}
Note: The general idea of this proof is to construct a family of open sets (the ``level sets'') which define our continuous function.

\nline

\textit{Step 1.} Let $P = \mathbb{Q} \cap [0,1] \subseteq \mathbb{R}$. Pick an ordering of $P$ so that $P = \{0, 1, p_2, p_3, \dots\}$ (note that we might not have $p_2 < p_3$ in the usual ordering of the rationals; this is an arbitrary order that we have \textit{imposed} upon the rationals in this interval). Our goal is to define, for each $p \in P$, an open set $U_p$ such that if $p < q$, then $\overline{U}_p \subseteq U_q$. To make our inductive case ``work'', define $P_n = \{0, 1, p_2, \cdots, p_n\}$ (i.e., all rational numbers up to the $n$-th rational).

\nline

For our base case, set $U_1 = X - B$. Since $A$ and $B$ are disjoint and $B$ is closed, this means that $A \subseteq U_1$ where $U_1$ is an open set. Since $X$ is normal, we can pick an open set $U_0$ such that $A \subseteq U_0 \subseteq \overline{U}_0 \subseteq U_1$ (by Lemma 31.1(b)). For our inductive case, suppose that each $U_p$ is constructed so that $\overline{U}_p \subseteq U_q$ when $p < q$ for all $p \in P_n$. We want to define $U_r$ for $r = p_{n+1}$ (i.e., for the \textit{next} rational number in the sequence). Note that because $P_n$ is a \textit{finite} set of rational numbers, it is well-ordered (every nonempty subset has a least element). This means that, because $0$ is the minimum element of $P_{n+1}$, $1$ is the maximum element of $P_{n+1}$, and $r$ is neither of those elements (we have already handled our base cases), $r$ must have an immediate predecessor and successor in $P_{n+1}$ (with respect to the standard ordering on the rationals). Define $p$ to be the immediate predecessor of $r$ and $q$, the immediate successor. By inductive hypothesis, we know that $U_p \subseteq \overline{U}_p \subseteq U_q$ (since $p, q \in P_n$). Since $X$ is normal, we can find an open set $U_r$ such that $\overline{U}_p \subseteq U_r \subseteq \overline{U}_r \subseteq U_q$.

\nline

Having inductively defined our open sets in this manner, we claim that the original condition (if $p < q$, $\overline{U}_p \subseteq U_q$) holds for all points in $P_{n+1}$. Why? If $p, q \in P_n$, then we apply our inductive hypothesis. Otherwise, we have $p=r$ and $q = s$ for some $s \in P_n$. If $r \leq s$, then $r < q \leq s$ (where $q$ is the immediate successor of $r$). We have already established that $\overline{U}_r \subseteq U_q$ and we know that $\overline{U}_q \subseteq U_s$ by our inductive hypothesis, so $\overline{U}_r \subseteq U_s$ as desired. The proof for the case where $r > s$ is similar.

\nline

\textit{Step 2.} We have now defined $U_p$ for all rational numbers $p \in [0,1]$. We can extend this definition to all rationals numbers in $\mathbb{R}$ by defining
\[
\begin{align}
    U_p &= \varnothing \quad \text{if} \; p < 0, \\
    U_p &= X \quad \text{if} \; p > 1.
\end{align}
\]
Why is this a valid definition? If $p < 0$, $\overline{U}_p = \varnothing$ which is clearly contained in every other set. If $q > 1$, then $U_q$ contains $\overline{U}_1$ because $U_q = X$.

\nline

\textit{Step 3.} Given $x \in X$, define $S_x = \{p \in \mathbb{Q} : x \in U_p\}$. In English, this is the set of rational numbers whose corresponding open sets contain $x$. Note that $S_x$ never contains any number less than $0$ because if $p < 0$, then we have defined $U_p = \varnothing$. Note also that $S_x$ contains \textit{every} number greater than $1$ because if $p > 1$, then $U_p = X$ and clearly $x \in X$ for all $x$. Hence, $S_x$ is bounded below by $0$ and nonempty so we can apply the completeness property of the real numbers to make the definition: $f(x) = \inf S_x$.

\nline

\textit{Step 4.} We need to show that $f(x) = 0$ for each $x \in A$ and $f(x) = 1$ for each $x \in B$, and finally, that $f$ is actually continuous. First, we want to show that $f(x) \in [0,1]$ for each $x \in X$. We know that $S_x$ is bounded from below by $0$ so $f(x) \geq 0$ for all $x$. We also know that if we suppose that $f(x) > 1$ for some $x \in \mathbb{R}$, this means that $x \notin U_p$ for any $p \in [0,1]$. But this is a contradiction because we can use $1 + \epsilon$. % WHAT??

\nline

If $x \in A$, then, since $A \subseteq U_0$, then $x \in U_p$ for all $p \geq 0$ (remember the $U_0$ is the ``smallest'' set in our nested sequence of sets). Hence, $\inf S_x = \inf \mathbb{Q}^+ = 0$.

\nline

On the other hand, if $x \in B$, then for all $p \leq 1$, $x \notin U_p$ (because $U_1$ was defined as the complement of $B$ and $U_1$ is the ``largest'' set in our sequence). Hence, $\inf S_x = \inf \{ q \in \mathbb{Q} : q > 1\} = 1$.

\nline

Finally, we want to show that $f$ is continuous. We will first prove two facts.

\begin{enumerate}
    \item If $x \in \overline{U}_r$, then $f(x) \leq r$. Why is this true? If $x \in \overline{U}_r$, then $x \in U_s$ for all $s > r$. Hence, $S_x$ contains all rational numbers which are greater than $r$ so $f(x) \leq r$.
    \item If $x \notin U_r$, then $f(x) \geq r$. Why is this true? If $x \notin U_r$, then $x \notin U_s$ for any $s < r$. Hence, $S_x$ does not contain any rational numbers less than $r$ so $f(x) \geq r$.
\end{enumerate}

Given a point $x_0 \in X$ and an open interval $(c,d) \subseteq \mathbb{R}$ containing $f(x_0)$, we want to show that there exists $U \ni x_0$ such that $f(U) \subseteq (c,d)$. % Supposing that this is actually a reasonable interpretation of normality for a function from an arbitrary topological space to the real numbers, let's continue... lol
Pick an arbitrary $x_0 \in X$. By density of the rationals, we can find $p, q \in \mathbb{Q}$ such that $c < p < f(x_0) < q < d$. We claim that if we set $U = U_q - \overline{U}_p$, then $x_0 \in U$ and $f(U) \subseteq (c,d)$.

\nline

Why is the first statement true? We know that $f(x_0) < q$ (by force), so by the contrapositive of fact (2) from above, $x_0 \in U_q$. On the other hand, since $f(x_0) > p$, by the contrapositive of fact (1) from above, $x_0 \notin \overline{U}_p$. Hence, $x_0 \in U = U_q - \overline{U}_p$. Why is the second statement true? If we pick any $x \in U$, we want to show that $f(x) > c$ and $f(x) < d$. If $x \in U$, then $x \in U_q$ so $f(x) \leq q < d$ (by fact (1)). However, we also know that $x \notin \overline{U}_p$ so $f(x) \geq p > c$ (by fact(2)). Hence, we have proven that $f$ is continuous.
\end{proof}

\subsection{The Urysohn Metrization Theorem}

\begin{theorem}[34.1, the Urysohn Metrization Theorem]
If $X$ is a regular, second-countable topological space, then $X$ is metrizable.
\end{theorem}

\begin{proof}
Suppose that we have already proven Theorem 34.2 and the corresponding lemma (both below). Since $X$ is regular and second-countable, there exists an embedding of $X$ in $\mathbb{R}^\omega$. We know that $\mathbb{R}^\omega$ is metrizable and since subspaces of metric spaces are also metric spaces and $F(X) \subseteq \mathbb{R}^\omega$, this implies that $X$ is metrizable.
\end{proof}

\begin{theorem}[34.2, the embedding theorem]
If $X$ is a regular, second-countable space, then there exists a continuous function $F: X \to \mathbb{R}^\omega$ which is a homeomorphism onto its image (an \textbf{embedding}).
\end{theorem}

\begin{proof}
\textit{Step 1.} We want to construct a countable collection of continuous functions $f_n : X \to [0,1]$ such that, given any point $x_0 \in X$ and an open set $U \ni x_0$, there exists $n$ such that $f_n(x_0) > 0$ and $f_n(y) = 0$ for all $y \notin U$. (Note: This is called a ``bump'' function.)

\nline

Let $\{B_n\}$ be a countable basis for $X$ ($X$ is second-countable). For each pair of indices $n, m$ such that $\overline{B}_n \subseteq B_m$, apply the Urysohn lemma to choose a continuous function $g_{n,m}: X \to [0,1]$ such that $g_{n,m}(\overline{B}_n) = 1$ and $g_{n,m}(X - B_m) = 0$. (Note that this is a valid application of the lemma because $\overline{B}_n$ is closed and $X - B_m$ is closed as the complement of an open set; we also know that $\overline{B}_n \subseteq B_m$ so these sets are disjoint). We claim that the collection of functions $g_{n,m}$, re-indexed as $f_n$, works as our countable collection. Why? Given any point $x_0 \in X$ and a neighborhood $U \ni x_0$, there exists a basis element $B_m$ such that $x_0 \in B_m \subseteq U$. Due to regularity, we can choose $B_n$ such that $x_0 \in B_n$ and $\overline{B}_n \subseteq B_m$ (by Lemma 31.1(a)). Let's now re-index $g_{n,m}$ as $f_n$.

\nline

\textit{Step 2.} We want to construct an embedding $F : X \to \mathbb{R}^\omega$. We claim that
\[
F(x) = (f_1(x), f_2(x), \dots )
\]
where each $f_i$ comes from our countable collection of continuous functions from above works as an embedding. Why? We need to show that $F$ is continuous, bijective, and that $F^{-1}$ is continuous as well.

\nline

First note that $F$ is continuous because each $f_i$ is continuous (by definition) and the product of continuous functions is continuous. Next, to show that $F$ is injective, suppose that $x \neq y$. By regularity, we can choose $U \ni x$ such that $y \notin U$. Then, by construction, there exists $n$ such that $f_n(x) > 0$ and $f_n(X - U) = 0$. Since $y \in X - U$, this implies that $f_n(y)$. We see that $F(x) = F(y)$ (because if the coordinate functions differ, the functions as a whole must have different values). Hence, $F$ is injective.

\nline

We also know that $F$ is necessarily surjective because it is a mapping onto its image. Finally, we want to show that $F$ is an open map (equivalent to showing that $F^{-1}$ is continuous). Suppose $U \subseteq X$ is open; we want to show that $F(U)$ is open in the image of $X$ under $F$. Pick a point $z_0 \in F(U)$. Define $Z = F(X)$. Let $z_0 = F(x_0)$ where $x_0 \in U$. By the construction from above, we can choose an index $N$ such that $f_N(x_0) > 0$ and $f_N(X - U) = 0$. Let $W = \{z \in Z : z_N > 0\}$ (i.e., the $n$-th coordinate of $z$ is positive). Note that $W$ is open in the subspace topology on $Z \subseteq \mathbb{R}^\omega$. We know that $z_0 \in W$ because the $n$-th coordinate of $z_0$ is $f_N(x_0) > 0$ by definition. We also know that $W \subseteq F(U)$ because if we pick any point $z \in W$, then we know that $z_N > 0$ so $f_N(x) > 0$ which implies that $x \in U$ so $z \in F(U)$. We conclude that $F(U)$ is open in $Z$.
\end{proof}

\begin{lemma}
$\mathbb{R}^\omega$ with the product topology is metrizable.
\end{lemma}

\begin{proof}
Define a metric $d$ on $\mathbb{R}^\omega$ where
\[
d(\textbf{x}, \textbf{y}) = \sup \left\{ \frac{d(x_i, y_i)}{i} \right\}.
\]
\end{proof}

\subsection{The Tynchonoff Theorem}

\begin{theorem}[37.3, the Tynchonoff Theorem]
If $J$ is \textit{any} indexing set and for all $\alpha \in J$, $X_\alpha$ is a compact topological space, then
\[
\prod_{\alpha \in J} X_\alpha
\]
is also compact in the product topology.
\end{theorem}

\subsection{The Stone-Cech Compactification}

\textit{Recall} that a \textbf{compactification} of a space $X$ is a compact Hausdorff space $Y$ containing $X$ such that $\overline{X} = Y$. Two compactifications $Y_1$ and $Y_2$ of $X$ are \textbf{equivalent} if there exists a homeomorphism between them that acts as the identity on $X$.

\nline

A \textbf{completely regular} space $X$ has the property that given $x \in X$ and a closed set $\overline{A} \subseteq X$ such that $x \notin A$, there exists a continuous function $f : X \to [0,1] \subseteq \mathbb{R}$ such that $f(x) = 0$ and $f(A) = 1$. (Note that this condition is stronger than regularity).

\nline

Take $X$ to be a completely regular space. Let $B$ be the set of all functions $f : X \to \mathbb{R}$ which are continuous and bounded (i.e., there exist $a, b \in \mathbb{R}$ such that $a < f(x) < b$ for all $x \in X$). Consider the map $F : X \to \prod_{f \in B} [a_f, b_f]$ which maps $F(X) = (F(X))_f$ where $a_f$ is the infimum (lower bound) and $b_f$ the supremum (upper bound) of the values taken by $f(X)$. Note that $F$ is indexed by the functions $f \in B$.

\nline

We claim that $F$ is an \textbf{embedding} (i.e., a homeomorphism onto its image). It is continuous because a function is continuous if and only if each of its coordinate functions are continuous. Why is $F$ injective? Since $X$ is completely regular, if we have $x,y$ such that $x \neq y$, then we can find $f : X \to [0,1]$ such that $f(x) = 0$ and $f(y) = 1$ (because of the definition of complete regularity and the fact that points are closed in completely regular spaces). This means that $f(x) \neq f(y)$ so $F(x) \neq F(y)$. Note that all functions are surjective onto their image. Finally, the proof that the inverse image of $F$ is surjective is the same as the proof of Theorem 34.2 (a regular, second-countable space embeds inside $\mathbb{R}^\omega$).

\nline

The \textbf{Stone-Cech compactification} of $X$ is the closure of $F(X) \subseteq \prod_{f \in B} [a_f, b_f]$. Note that $\prod_{f \in B} [a_f, b_f]$ is compact due to the Tynchonoff theorem and since closed subsets of compact spaces are also compact, this compactification ``works'' the way we would expect.

\nline

Example: compactifications of $X = (0,1)$

\begin{itemize}
    \item The one-point compactification $X^*$ is homeomorphic to $S^1$.
    \item The two-point compactification is $[0,1]$.
    \item The closure of the topologist's sine curve is a compactification (adds the point $\{1\}$ at one end and the interval $[-1,1]$ at the other).
\end{itemize}

Loosely, the Stone-Cech compactification is the ``largest'' compactification and the one-point compactification is the ``smallest''.

\section{The Fundamental Group}

\subsection{Path Homotopy}

If $f$ and $g$ are continuous maps of the space $X$ into the space $Y$, $f$ is \textbf{homotopic} to $g$ if here exists a continuous map $H : X \times [0,1] \to Y$ such that
\[
H(x, 0) = f(x) \quad \text{and} \quad H(x, 1) = g(x).
\]

Notation: If $f$ is homotopic to $g$, we write $f \simeq g$.

\nline

We can think of homotopy as giving us an ``interpolating'' family of maps in the image space which connects the images of $f$ and $g$. We say that $H$ ``interpolates'' between $f$ and $g$ via continuous functions.

\nline

Two paths $f, g : [0,1] \to X$ are \textbf{path-homotopic} if $f(0) = g(0) = x_0$, $f(1) = g(1) = x_1$, and there exists a continuous map $H : I \times I \to X$ such that
\[
\begin{align}
    H(s, 0) = f(s) \quad &\text{and} \quad H(s, 1) = g(s) \\
    H(0, t) = x_0 \quad &\text{and} \quad H(1, t) = x_1,
\end{align}
\]
for all $s \in [0,1]$ and $t \in [0,1]$, then $H$ is a \textbf{path homotopy} between $f$ and $g$.

\nline

Notation: If $f$ and $g$ are path-homotopic, then we write $f \simeq_p g$.

\nline

Loosely, this means that $f$ and $g$ are homotopic via a homotopy that fixes the endpoints of both paths.

\begin{lemma}[51.1]
The relations $\simeq$ and $\simeq_p$ are equivalence relations on paths.
\end{lemma}

\begin{proof}
We will check that $\simeq$ is an equivalence relation.

\begin{enumerate}
    \item Reflexivity? We need to show that $f : X \to Y$ is homotopic to itself. Let $H(x, t) = f(x)$ (``the homotopy that doesn't do anything'').
    \item Symmetry? We need to show that if $f \simeq g$, then $g \simeq f$. Let $H: X \times I \to Y$ be a homotopy from $f$ to $g$ where
    \[
    H(x, 0) = f(x) \quad \text{and} \quad H(x, 1) = g(x).
    \]
    Define the homotopy $K(x, t) = H(x, 1-t)$. This reverses the directionality of $H$. We check:
    \[
    K(x, 0) = H(x, 1) = g(x) \quad \text{and} \quad K(x,1) = H(x, 0) = f(x).
    \]
    \item Transitivity? Suppose that $f \simeq g$ via $H$ and $g \simeq j$ via $K$. This means that
    \[
    \begin{align}
        H(x, 0) = f(x) \quad &\text{and} \quad H(x, 1) = g(x) \\
        K(x, 0) = g(x) \quad &\text{and} \quad K(x, 1) = j(x).
    \end{align}
    \]
    Define
    \[
    M(x, t) = \begin{cases}
    H(x, 2t) \quad \text{if} \; t \in [0, \frac{1}{2}] \\
    K(x, 2t - 1) \quad \text{if} \; t \in [\frac{1}{2}, 1].
    \end{cases}
    \]
    Note that this is akin to saying ``go along the first homotopy twice as fast and then go along the second homotopy twice as fast''. This is indeed a continuous map due to the pasting lemma.
\end{enumerate}
\end{proof}

Let $X, Y$ be topological spaces. Then $f : X \to Y$ is a \textbf{homotopy equivalence} if there exists a continuous map $g : Y \to X$ such that $f \circ g \simeq id_Y$ and $g \circ f \simeq_X$. We say that $X, Y$ are \textbf{homotopy equivalent} spaces.

\nline

Examples:

\begin{itemize}
    \item $\mathbb{R}^n$ is homotopy equivalence to $\{0\}$.
    \item In general, a space that is homotopy equivalent to a single space is \textbf{contractible}.
    \item Let $X = \mathbb{R}^2 - \{0\}$ (the punctured plane) and $Y = S^1$. Then $X, Y$ are homotopy equivalent. Why is this the case? Define maps
    \[
    \begin{align}
        f : S^1 \to \mathbb{R}^2 - \{0\} \quad &\text{where} \quad f(x, y) = (x,y) \\
        g : \mathbb{R}^2 - \{0\} \to S^1 \quad &\text{where} \quad g(x, y) = \frac{(x,y)}{||(x,y)||}.
    \end{align}
    \]
    Note that $g \circ f : S^1 \to S^1$ is $g(f(x,y)) = g(x,y) = \frac{(x,y)}{||(x,y)||}$ which is equivalent to the identity on $S^1$ because all elements in $S^1$ are of norm $1$. On the other hand, $f \circ g : \mathbb{R}^2 - \{0\} \to \mathbb{R}^2 - \{0\}$ is $f(g(x,y)) = f(\frac{(x,y)}{||(x,y)||} = \frac{(x,y)}{||(x,y)||}$. This is clearly \textit{not} equivalent to the identity on $\mathbb{R}^2 - \{0\}$ but we claim that it is homotopic to it. Define the homotopy $H : \mathbb{R}^2 - \{0\} \times I \to \mathbb{R}^2 - \{0\}$ by
    \[
    H((x,y), t) = \frac{t(x,y)}{||(x,y)||} + (1 - t)(x,y).
    \]
    We can check that: (1) this is indeed a homotopy, (2) the output of this homotopy is always nonzero, and (3) this homotopy is continuous.
\end{itemize}

If $\gamma: I \to X$ and $\rho: I \to X$ are continuous paths such that $\gamma(1) = \rho(0)$ (share endpoints), define the path $\rho * \gamma$ to be
\[
(\rho \times \gamma)(t) = \begin{cases}
\gamma(2t) \quad \text{if} \; t \in [0,\frac{1}{2}] \\
\rho(2t - 1) \quad \text{if} \; t \in [\frac{1}{2}, 1].
\end{cases}
\]
We can think of this as ``path concatenation''.

\nline

We need to check certain things about this new operation defined on paths: (1) well-defined, (2) associativity, (3) existence of identity, and (4) existence of inverses.

\begin{theorem}[51.2]
The operation $*$ defined on path homotopy classes has the following properties:
\begin{enumerate}
    \item The operation $*$ is well defined. That is, if $\rho \simeq_p \rho'$ and $\gamma \simeq_p \gamma'$, then $\rho * \gamma \simeq_p \rho' * \gamma'$.
    \item The operation $*$ is associative.
    \item The operation $*$ has an identity. Given $x \in X$, define $e_x : I \to X$ to be the constant path with $e_x(t) = x$ for all $t$. If $\gamma(0) = x_0$ and $\gamma(1) = x_1$, then
    \[
    [\gamma] * [e_{x_1}] = [\gamma] \quad \text{and} \quad [e_{x_0}] * [\gamma] = [\gamma].
    \]
    \item For each path $\gamma : I \to X$, $\gamma$ has an inverse with respect to $*$.
\end{enumerate}
\end{theorem}

\begin{proof}
We will check each of these axioms for the cases where $*$ is defined on loop homotopy classes.

\nline

We first check that $*$ is well-defined. Suppose that $\rho$ and $\rho'$ are homotopic under $H$ and $\gamma$ and $\gamma'$ are homotopic under $K$. Then define
\[
M(s, t) = \begin{cases}
H(2s, t) \quad \text{for} \; s \in [0,\frac{1}{2}] \\
K(2s - 1, t) \quad \text{for} \; s \in [\frac{1}{2}, 1].
\end{cases}
\]
We can think of this as ``sticking the homotopies together''. This confirms that
\[
[\rho] * [\gamma] = [\rho * \gamma]
\]
is a well-defined operation.

\nline

Next, we check that $*$ is associative on loop homotopy classes. To do this, we want to show that $[\delta] * ([\rho] * [\gamma]) = ([\delta] * [\rho]) * [\gamma]$. (Note: In general, multiplication of paths is \textit{not} associative, but multiplication of path \textit{homotopy classes} is.) The proof is given by the following image.

\begin{center}
\includegraphics[scale=0.3]{Julie/assoc.jpg}
\end{center}

The proof of (3), existence of identity, relies on a similar technique (``wait at $x_0$ for half-time, then move along the path'').

\nline

For the proof of (4), note that the inverse path $\overline{\gamma(t)} = \gamma(1 - t)$. We claim that there exists a homotopy between $\overline{\gamma} * \gamma$ and the constant path, and a homotopy between $\gamma * \overline{\gamma}$ and the constant path. The homotopy is simply ``start out at $x_0$, move an infinitesimal distance along $\gamma$ and then go back; repeat adding incremental distance each time until you have traveled the whole path and back''.
\end{proof}

\subsection{The Fundamental Group}

Let $X$ be a space and let $x_0 \in X$. A path in $X$ that begins and ends at $x_0$ is a \textbf{loop} based at $x_0$. The set of path homotopy classes of loops based at $x_0$ under the operation $*$ is the \textbf{fundamental group} of $X$ based at $x_0$. We denote it
\[
\pi_1(X, x_0) := \{ \; \text{loops $\gamma$ in $X$ based at $x_0$} \; \} / \simeq_p.
\]

Question: To what extent does the fundamental group of a space depend on the choice of basepoint $x_0$?

\nline

Let $X$ be a topological space, $x_0, x_1 \in X$ be two points, and assume there exists a path $\alpha : I \to X$ such that $\alpha(0) = x_0$ and $\alpha(1) = x_1$. (Note that this implies $x_0, x_1$ are in the same path component!)

\nline

Goal: Define a homomorphism from $\pi_1(X, x_0) \to \pi_1(X, x_1)$. Consider the map
\[
\hat{\alpha} : \pi_1(X, x_0) \to \pi_2(X, x_1) \quad \text{where} \quad \hat{\alpha}(\gamma) = \overline{\alpha} * \gamma * \alpha$.
\]

\begin{center}
    \includegraphics[scale=0.5]{Julie/thing.png}
\end{center}

\begin{theorem}[52.1]
The map $\hat{\alpha}$ is a well-defined group isomorphism from $\pi_1(X, x_0)$ to $\pi_1(X, x_1)$.
\end{theorem}

\begin{proof}
We first check that if $\gamma \simeq_p \gamma'$, then $\overline{\alpha} * \gamma * \alpha \simeq_p \overline{\alpha} * \gamma' * \alpha$. Define $H: I \times I \to X$ where $H(t, 0) = H_0(t) = \gamma(t)$ and $H(t, 1) = H_1(t)  \gamma'(t)$. Then the map $K(t, s) = \overline{\alpha} * (H_s * \alpha) (t)$ is a homotopy between $\overline{\alpha} * \gamma * \alpha$ and $\overline{\alpha} * \gamma' * \alpha$.

\nline

Next, we check that $\hat{\alpha}$ is a homomorphism.

\[
\begin{align}
    \hat{\alpha}(\gamma * \rho) &= \overline{\alpha}* ((\gamma * \rho) * \alpha) \\
    &\simeq_p \overline{\alpha} * \gamma * c_{x_1} * \rho * \alpha \\
    &\simeq_p \overline{\alpha} * \gamma * \alpha * \overline{\alpha} * \rho * \alpha \\
    &= \hat{\alpha}(\gamma) * \hat{\alpha} (\rho).
\end{align}
\]
Define an inverse homomorphism $\hat{\overline{\alpha}}(\gamma) = \alpha * \gamma * \overline{\alpha}$. Then
\[
\begin{align}
    \hat{\overline{\alpha}}(\hat{\alpha}(\gamma)) &= \hat{\overline{\alpha}}(\overline{\alpha} * \gamma * \alpha) \\
    &= \alpha * \overline{\alpha} * \gamma * \alpha * \overline{\alpha} \\
    &\simeq_p c_{x_1} * \gamma * c_{x_0} \\
    &= \gamma.
\end{align}
\]
\end{proof}

\begin{corollary}[52.2]
If $X$ is path connected and $x_0, x_1 \in X$, then $\pi_1(X, x_0) \cong \pi_1(X, x_1)$.
\end{corollary}

Some notes about this fact: If $C \subseteq X$ is the path-component of $X$ containing $x_0$, then $\pi_1(C, x_0) = \pi_1(X, x_0)$ because all loops must be within $C$. If $X$ is path connected, then all groups $\pi_1(X, x_0)$ for each $x_0 \in X$ are isomorphic; however, they are not necessarily isomorphic via the same isomorphism.

\nline

A space $X$ is \textbf{simply-connected} if it is path connected and if $\pi_1(X, x_0)$ is the trivial (single-element) gorup for some $x_0 \in X$ and hence, for each $x_0 \in X$.

\nline

Let $X$ be a topological space with $x_0 \in X$. Let $Y$ be another topological space and $f : X \to Y$ a continuous map. Let $y_0 = f(x_0)$. The \textbf{induced homomorphism} $f_* : \pi_1(X, x_0) \to \pi_1(Y, y_0)$ is defined by $f_*([\gamma]) = [f \circ \gamma]$.

\nline

Why is this well-defined? We need to check that if $\gamma \simeq_p \gamma'$, then $f \circ \gamma \simeq_p f \circ \gamma'$. If $H : I \times I \to X$ is the homotopy from $\gamma$ to $\gamma'$, then $f \circ H$ is a homotopy from $f \circ \gamma$ to $f \circ \gamma'$. Why is this a homomorphism? Note that
\[
\begin{align}
    f_*(\gamma * \rho) &= f \circ (\gamma * \rho) \\
    &= \begin{cases}
    f(\rho(2t)) \quad \text{if} \; t \in [0,\frac{1}{2}] \\
    f(\gamma(2t - 1)) \quad \text{if} \; t \in [\frac{1}{2},1].
    \end{cases}
\end{align}
\]
But note that this last expression is exactly $(f \circ \gamma) * (f \circ \rho)$.

\begin{theorem}[52.4]
If $f : X \to Y$, $g : Y \to Z$ are continuous such that $f(x_0) = y_0$ and $g(y_0) = z_0$, then $(g \circ f)_* = g_* \circ f_*$. If $i$ is the identity map on $X$, then $i_*$ is the identity homomorphism.
\end{theorem}

\begin{proof}
Note that
\[
\begin{align}
    (g \circ f)_*([\gamma]) &= [(g \circ f) \circ \gamma] \\
    (g_* \circ f_*)([\gamma]) &= g_*(f_*([\gamma])) = g_*([f \circ \gamma]) = [g \circ (f \circ \gamma)].
\end{align}
\]
\end{proof}

\begin{corollary}[52.5]
If $f : X \to Y$ is a homeomorphism, then the induced map $f_*: \pi_1(X, x_0) \to \pi_1(Y, y_0)$ is an isomorphism.
\end{corollary}

\begin{proof}
Since $f$ is a homeomorphism, let $f^{-1}$ be its inverse. Then $f^{-1}_* \circ f_* = (f^{-1} \circ f)_* = i_*$ where $i$ is the identity on $X$ and $f_* \circ f^{-1}_* = (f \circ f^{-1})_* = j_*$ where $j$ is the identity on $Y$. But this means that $f^{-1}_*$ is the inverse homomorphism of $f_*$.
\end{proof}

A couple of important things about homotopy theory:

\begin{enumerate}
    \item Homotopy equivalence is a ``weaker'' notion of equivalence than homeomorphism (homeomorphism implies homotopy equivalence but not necessarily the converse).
    \item Note that homotopy equivalence does \textit{not} necessarily preserve the ``nice'' properties that are invariant under homeomorphism.
    \item If $X$ and $Y$ are homotopy equivalent, then $\pi_1(X, x_0) \cong \pi_1(Y, y_0)$ for some $y_0 \in Y$. This means that if $X, Y$ do \textit{not} have isomorphic fundamental groups, then they are not homotopy equivalent and thus not homeomorphic.
\end{enumerate}

\textit{Note.} If $X$ is contractible, then $\pi_1(X, x_0)$ is the trivial group. Why? Recall that the fundamental group is homotopy invariant so since contractible means homotopy equivalent to a single point and a single point (obviously) has trivial fundamental group, we have the result. This implies that the fundamental group of any Euclidean space is trivial.



\end{flushleft}
\end{document}
